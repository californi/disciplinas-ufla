\documentclass[aspectratio=169]{beamer}

% Configurações do tema
\usetheme{Berkeley}
\usecolortheme{spruce}
\setbeamertemplate{navigation symbols}{}
\setbeamertemplate{footline}[frame number]

% Configurações específicas para o sumário
\setbeamertemplate{section in toc}[sections numbered]
\setbeamertemplate{subsection in toc}[subsections numbered]

% Configuração do logo da UFLA
\setbeamertemplate{headline}{%
  \begin{beamercolorbox}[wd=\paperwidth,ht=1.2cm,dp=0.2cm]{section in head/foot}%
    \begin{center}
      \vspace{0.1cm}
      \includegraphics[height=1.0cm]{../../Image/ufla-logo.PNG}
    \end{center}
  \end{beamercolorbox}%
}

% Configuração do título centralizado
\setbeamertemplate{frametitle}{%
  \begin{beamercolorbox}[wd=\paperwidth,ht=1.2cm,dp=0.3cm]{frametitle}%
    \begin{center}
      \insertframetitle
    \end{center}
  \end{beamercolorbox}%
}

% Pacotes necessários
\usepackage[utf8]{inputenc}
\usepackage[brazilian]{babel}
\usepackage{amsmath}
\usepackage{amsfonts}
\usepackage{amssymb}
\usepackage{graphicx}
\usepackage{hyperref}
\usepackage{multicol}

% Configurações de cores
\definecolor{azul}{RGB}{0,51,102}
\definecolor{azulclaro}{RGB}{51,102,153}
\setbeamercolor{title}{fg=white,bg=azul}
\setbeamercolor{frametitle}{fg=white,bg=azulclaro}
\setbeamercolor{section title}{fg=white,bg=azul}

% Informações do documento
\title{Projeto de Software}
\subtitle{GCT088 - Aula 1.1 - Apresentação da Disciplina}
\author{Bento Rafael Siqueira}
\institute{Universidade Federal de Lavras (UFLA)}
\date{\today}

\begin{document}

% Slide de título
\begin{frame}
\titlepage
\end{frame}

% Slide de sumário
\begin{frame}{Sumário}
\tableofcontents
\end{frame}

% Seção 1: Informações Gerais
\section{Informações Gerais}

\begin{frame}{Informações Gerais}
\begin{itemize}
    \item \textbf{Disciplina:} Projeto de Software
    \item \textbf{Código:} GCT088
    \item \textbf{Carga Horária:} 68 horas (34 Teórica, 34 Prática) 17 semanas
    \item \textbf{Aulas Semanais:} 4 aulas
    \item \textbf{Pré-requisitos:} Programação Orientada a Objetos
\end{itemize}
\end{frame}

% Seção 2: Objetivos
\section{Objetivos}

\begin{frame}{Objetivos Gerais}
\begin{itemize}
    \item Compreender os fundamentos do projeto e desenvolvimento de software
    \item Aplicar técnicas de análise e projeto orientado a objetos
    \item Desenvolver sistemas seguindo boas práticas de engenharia de software
    \item Estimular o trabalho em equipe e gestão de projetos
\end{itemize}
\end{frame}

\begin{frame}{Objetivos Específicos}
\begin{itemize}
    \item Dominar técnicas de obtenção e análise de requisitos
    \item Aplicar princípios de projeto e arquitetura de software
    \item Implementar padrões de projeto e boas práticas
    \item Desenvolver modelos UML para análise e projeto
    \item Gerenciar mudanças de requisitos e flexibilidade de software
\end{itemize}
\end{frame}

% Seção 3: Ementa
\section{Ementa}

\begin{frame}{Ementa}
A disciplina aborda os conceitos fundamentais de projeto de software, incluindo:
\begin{itemize}
    \item Introdução a objetos e conceitos básicos de orientação a objetos
    \item Projeto de aplicações e obtenção de requisitos
    \item Análise e modelagem usando UML
    \item Flexibilidade de software e gestão de mudanças
    \item Arquitetura de software e princípios de projeto
    \item Padrões de projeto e testes
    \item Ciclo de vida em Análise e Projeto Orientado a Objetos
    \item Desenvolvimento web e projetos práticos
\end{itemize}
\end{frame}

% Seção 4: Conteúdo Programático
\section{Conteúdo Programático}

\begin{frame}{Conteúdo Programático}
\begin{columns}
\begin{column}{0.5\textwidth}
\begin{enumerate}
    \item Apêndice - Introdução a Objetos
    \item Projeto de Aplicações
    \item Obtenção de Requisitos
    \item Mudança de Requisitos
    \item Análise
    \item Flexibilidade de software - Parte 1
    \item Flexibilidade de software - Parte 2
    \item Grandes problemas
    \item Arquitetura
    \item Princípios de Projeto
\end{enumerate}
\end{column}
\begin{column}{0.5\textwidth}
\begin{enumerate}
    \setcounter{enumi}{10}
    \item Repetição e Testes
    \item Ciclo de vida em A\&POO
    \item Modelos de Contexto
    \item Modelos de interação
    \item Modelos estruturais
    \item Modelos comportamentais
    \item Engenharia dirigida por modelos
    \item Padrões de Projetos
    \item Desenvolvimento Web
    \item Projeto - ONG de Animais
\end{enumerate}
\end{column}
\end{columns}
\end{frame}

% Seção 5: Bibliografia
\section{Bibliografia}

\begin{frame}{Bibliografia Básica}
\begin{itemize}
    \item \textbf{PRESSMAN, Roger S.} Engenharia de Software: uma abordagem profissional. 8. ed. Porto Alegre: AMGH, 2016.
    \item \textbf{SOMMERVILLE, Ian.} Engenharia de Software. 10. ed. São Paulo: Pearson, 2019.
    \item \textbf{FOWLER, Martin.} UML Essencial: um breve guia para o padrão. 3. ed. Porto Alegre: Bookman, 2005.
    \item \textbf{GAMMA, Erich et al.} Padrões de projeto: soluções reutilizáveis de software orientado a objetos. Porto Alegre: Bookman, 2011.
\end{itemize}
\end{frame}

\begin{frame}{Bibliografia Complementar}
\begin{itemize}
    \item \textbf{BOOCH, Grady et al.} UML: guia do usuário. 2. ed. Rio de Janeiro: Campus, 2006.
    \item \textbf{LARMAN, Craig.} Utilizando UML e padrões: uma introdução à análise e ao projeto orientados a objetos e ao processo unificado. 3. ed. Porto Alegre: Bookman, 2007.
    \item \textbf{COCKBURN, Alistair.} Desenvolvimento ágil de software. São Paulo: Pearson, 2007.
\end{itemize}
\end{frame}

% Seção 6: Critérios de Avaliação
\section{Critérios de Avaliação}

\begin{frame}{Critérios de Avaliação}
\begin{itemize}
    \item \textbf{Prova 1 (P1):} 25\%
    \item \textbf{Prova 2 (P2):} 25\%
    \item \textbf{Aula de Atividade (AA):} 20\%
    \item \textbf{Projeto em Grupo (PG):} 30\% (3 a 4 pessoas)
\end{itemize}

\vspace{0.5cm}
\textbf{Nota Final = P1 + P2 + AA + PG}
\end{frame}

% Seção 7: Cronograma Geral
\section{Cronograma Geral}

\begin{frame}{Cronograma Geral}
\begin{itemize}
    \item \textbf{Semanas 1-2:} Introdução a Objetos e Projeto de Aplicações
    \item \textbf{Semanas 3-4:} Obtenção e Mudança de Requisitos
    \item \textbf{Semanas 5-6:} Análise e Flexibilidade de Software
    \item \textbf{Semanas 7-8:} Grandes Problemas e Arquitetura
    \item \textbf{Semanas 9-10:} Princípios de Projeto e Testes
    \item \textbf{Semanas 11-13:} Modelos UML e Engenharia Dirigida por Modelos
    \item \textbf{Semanas 14-15:} Padrões de Projeto
    \item \textbf{Semanas 16-17:} Desenvolvimento Web e Projeto Final
\end{itemize}
\end{frame}

% Seção 8: Informações Adicionais
\section{Informações Adicionais}

\begin{frame}{Horário de Atendimento}
\begin{itemize}
    \item \textbf{Dia:} Quinta-feira
    \item \textbf{Horário:} 14:00 às 15:00
    \item \textbf{Email:} bento.siqueira@ufla.br
    \item \textbf{Local:} Sala PAV1-109
\end{itemize}
\end{frame}

\begin{frame}{Metodologia}
\begin{itemize}
    \item Aulas expositivas com exemplos práticos
    \item Exercícios em sala de aula
    \item Desenvolvimento de projetos práticos
    \item Trabalho em equipe
    \item Uso de ferramentas de modelagem (UML)
    \item Estudos de caso reais
\end{itemize}
\end{frame}

% Slide final
\begin{frame}{Dúvidas?}
\begin{center}
\Large
\textbf{Obrigado pela atenção!}

\vspace{1cm}
\textbf{Projeto de Software}

\vspace{0.5cm}
GCT088 - 2025
\end{center}
\end{frame}

\end{document}

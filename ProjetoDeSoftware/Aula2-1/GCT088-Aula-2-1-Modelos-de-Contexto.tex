\documentclass[aspectratio=169]{beamer}

% Configurações do tema
\usetheme{Berkeley}
\usecolortheme{spruce}
\setbeamertemplate{navigation symbols}{}
\setbeamertemplate{footline}[frame number]

% Configurações específicas para o sumário
\setbeamertemplate{section in toc}[sections numbered]
\setbeamertemplate{subsection in toc}[subsections numbered]

% Configuração do logo da UFLA
\setbeamertemplate{headline}{%
  \begin{beamercolorbox}[wd=\paperwidth,ht=1.2cm,dp=0.2cm]{section in head/foot}%
    \begin{center}
      \vspace{0.1cm}
      \includegraphics[height=1.0cm]{../../Image/ufla-logo.PNG}
    \end{center}
  \end{beamercolorbox}%
}

% Configuração do título centralizado
\setbeamertemplate{frametitle}{%
  \begin{beamercolorbox}[wd=\paperwidth,ht=1.2cm,dp=0.3cm]{frametitle}%
    \begin{center}
      \insertframetitle
    \end{center}
  \end{beamercolorbox}%
}

% Pacotes necessários
\usepackage[utf8]{inputenc}
\usepackage[brazilian]{babel}
\usepackage{amsmath}
\usepackage{amsfonts}
\usepackage{amssymb}
\usepackage{graphicx}
\usepackage{hyperref}
\usepackage{multicol}
\usepackage{tikz}

% Configurações de cores
\definecolor{azul}{RGB}{0,51,102}
\definecolor{azulclaro}{RGB}{51,102,153}
\definecolor{lightblue}{RGB}{173,216,230}
\definecolor{lightgreen}{RGB}{144,238,144}
\definecolor{lightcoral}{RGB}{240,128,128}
\definecolor{lightyellow}{RGB}{255,255,224}
\definecolor{lightpink}{RGB}{255,182,193}
\definecolor{lightgray}{RGB}{211,211,211}
\definecolor{orange}{RGB}{255,165,0}
\setbeamercolor{title}{fg=white,bg=azul}
\setbeamercolor{frametitle}{fg=white,bg=azulclaro}
\setbeamercolor{section title}{fg=white,bg=azul}

% Informações do documento
\title{Projeto de Software}
\subtitle{GCT088 - Aula 2.1 - Modelos de Contexto}
\author{Prof. Dr. Bento Rafael Siqueira}
\institute{Universidade Federal de Lavras\\Departamento de Ciência da Computação}
\date{\today}

\begin{document}

% Slide de título
\begin{frame}
\titlepage
\end{frame}

% Slide de sumário
\begin{frame}{Sumário}
\tableofcontents
\end{frame}

% Seção 1: Introdução
\section{Introdução aos Modelos de Contexto}

\begin{frame}{O que são Modelos de Contexto?}
\begin{itemize}
\item \textbf{Definição}: Representação do ambiente no qual o sistema opera
\item \textbf{Objetivo}: Mostrar como o sistema interage com elementos externos
\item \textbf{Escopo}: Define os limites do sistema e suas interfaces
\item \textbf{Importância}: Fundamental para compreender requisitos e arquitetura
\end{itemize}

\vspace{0.5cm}
\textbf{Exemplo Simples: Sistema de Login}
\begin{itemize}
\item \textbf{Sistema Central}: Sistema de Login
\item \textbf{Atores Externos}: Usuário, Banco de Dados, Sistema de Email
\item \textbf{Fluxos}: Credenciais $\rightarrow$ Validação $\rightarrow$ Notificação
\end{itemize}

\vspace{0.3cm}
\textbf{Fonte}: Sommerville, I. Engenharia de Software - Seção 5.1
\end{frame}

\begin{frame}{Por que são importantes?}
\begin{enumerate}
\item \textbf{Identificam atores externos} e suas interações
\item \textbf{Definem fronteiras claras} do sistema
\item \textbf{Facilitam comunicação} entre stakeholders
\item \textbf{Servem como base} para especificação de requisitos
\item \textbf{Orientam decisões} arquiteturais
\end{enumerate}

\vspace{0.5cm}
\textbf{Stakeholders do Sistema ONG de Animais:}
\begin{itemize}
\item \textbf{Clientes}: Adotantes de animais
\item \textbf{Desenvolvedores}: Equipe de desenvolvimento
\item \textbf{Administradores}: Gestores da ONG
\item \textbf{Voluntários}: Pessoas que ajudam na ONG
\item \textbf{Veterinários}: Profissionais de saúde animal
\item \textbf{Patrocinadores}: Apoiadores financeiros
\end{itemize}
\end{frame}

% Seção 2: Conceitos Fundamentais
\section{Conceitos Fundamentais}

\begin{frame}{Elementos dos Modelos de Contexto}
\begin{enumerate}
\item \textbf{Sistema Central}
\begin{itemize}
\item O sistema sendo desenvolvido
\item Representado como uma caixa preta
\item Foco nas interfaces, não na implementação
\end{itemize}

\item \textbf{Atores Externos}
\begin{itemize}
\item Pessoas, sistemas ou dispositivos externos
\item Interagem com o sistema central
\item Podem ser usuários, outros sistemas ou sensores
\end{itemize}
\end{enumerate}
\end{frame}

\begin{frame}{Interações nos Modelos de Contexto}
\begin{itemize}
\item \textbf{Fluxos de Dados}
\begin{itemize}
\item Comunicação entre sistema e atores externos
\item Podem ser bidirecionais
\item Representam troca de informações
\end{itemize}

\item \textbf{Tipos de Interação}
\begin{itemize}
\item Entrada de dados (usuário $\rightarrow$ sistema)
\item Saída de dados (sistema $\rightarrow$ usuário)
\item Consultas (sistema $\rightarrow$ banco de dados)
\item Notificações (sistema $\rightarrow$ email)
\end{itemize}
\end{itemize}
\end{frame}

\begin{frame}{Características dos Modelos de Contexto}
\begin{itemize}
\item \textbf{Visão de alto nível} - Sem detalhes de implementação
\item \textbf{Foco nas interfaces externas} - Como o sistema se conecta ao mundo
\item \textbf{Independente de tecnologia} - Não especifica linguagens ou plataformas
\item \textbf{Orientado a requisitos} - Base para identificação de funcionalidades
\item \textbf{Comunicativo} - Linguagem comum entre equipe e clientes
\end{itemize}
\end{frame}

\begin{frame}{Níveis de Modelagem de Contexto}
\begin{enumerate}
\item \textbf{Alto Nível (Visão Executiva)}
\begin{itemize}
\item Visão geral do sistema e ambiente
\item Foco nas principais interações
\item Apropriado para apresentações executivas
\end{itemize}

\item \textbf{Nível Médio (Visão Gerencial)}
\begin{itemize}
\item Detalhamento moderado das interfaces
\item Inclui principais fluxos de dados
\item Usado para planejamento de projeto
\end{itemize}

\item \textbf{Nível Detalhado (Visão Técnica)}
\begin{itemize}
\item Especificação detalhada das interfaces
\item Inclui protocolos e formatos de dados
\item Usado para desenvolvimento e integração
\end{itemize}
\end{enumerate}
\end{frame}

% Seção 3: Tipos de Modelos
\section{Tipos de Modelos de Contexto}

\begin{frame}{Classificação por Granularidade}
\begin{enumerate}
\item \textbf{Modelo de Contexto de Alto Nível}
\begin{itemize}
\item Visão geral do sistema e ambiente
\item Foco nas principais interações
\item Apropriado para apresentações executivas
\end{itemize}

\item \textbf{Modelo de Contexto Detalhado}
\begin{itemize}
\item Especificação detalhada das interfaces
\item Inclui protocolos e formatos de dados
\item Usado para desenvolvimento e integração
\end{itemize}
\end{enumerate}

\vspace{0.5cm}
\textbf{Exemplo de Granularidade:}
\begin{itemize}
\item \textbf{Alto Nível}: Usuário, BD, Email
\item \textbf{Detalhado}: Usuário(login, senha), BD(SQL, conexão), Email(SMTP, formato)
\end{itemize}
\end{frame}

\begin{frame}{Classificação por Domínio}
\begin{enumerate}
\item \textbf{Sistemas de Informação}
\begin{itemize}
\item Foco em usuários e dados
\item Interfaces com bancos de dados
\item Relatórios e consultas
\end{itemize}

\item \textbf{Sistemas Embarcados}
\begin{itemize}
\item Interação com sensores e atuadores
\item Controle de hardware
\item Tempo real
\end{itemize}

\item \textbf{Sistemas Web}
\begin{itemize}
\item Interfaces com navegadores
\item APIs e serviços
\item Escalabilidade
\end{itemize}
\end{enumerate}
\end{frame}

% Seção 4: Diagramas de Contexto
\section{Diagramas de Contexto}

\begin{frame}{Elementos Básicos dos Diagramas}
\textbf{Sistema Central:}
\begin{itemize}
\item Representado por um círculo ou retângulo
\item Nome do sistema no centro
\item Sem detalhes internos
\item Foco principal do diagrama
\end{itemize}

\vspace{0.3cm}
\textbf{Atores Externos:}
\begin{itemize}
\item Representados por retângulos ou figuras
\item Nome descritivo do ator
\item Posicionados ao redor do sistema
\item Interagem com o sistema central
\end{itemize}
\end{frame}

\begin{frame}{Fluxos de Dados - Características}
\textbf{Características dos Fluxos:}
\begin{itemize}
\item Setas indicando direção
\item Rótulos descritivos
\item Podem ser bidirecionais
\item Representam troca de informações
\end{itemize}

\vspace{0.5cm}
\textbf{Tipos de Interação:}
\begin{itemize}
\item \textbf{Entrada}: Dados recebidos pelo sistema
\item \textbf{Saída}: Dados enviados pelo sistema
\item \textbf{Bidirecional}: Consultas e respostas
\end{itemize}
\end{frame}

\begin{frame}{Fluxos de Dados - Importância}
\textbf{Importância dos Fluxos:}
\begin{itemize}
\item Mostram como o sistema se comunica
\item Identificam interfaces necessárias
\item Facilitam o entendimento do contexto
\end{itemize}

\vspace{0.5cm}
\textbf{Benefícios:}
\begin{itemize}
\item Comunicação clara entre componentes
\item Identificação de dependências
\item Base para especificação de interfaces
\end{itemize}
\end{frame}

\begin{frame}{Convenções de Diagramação - Organização}
\textbf{Organização Visual:}
\begin{itemize}
\item Sistema central no centro
\item Atores externos nas bordas
\item Fluxos identificados
\end{itemize}

\vspace{0.5cm}
\textbf{Objetivos:}
\begin{itemize}
\item Facilitar compreensão visual
\item Padronizar representação
\item Reduzir ambiguidades
\end{itemize}
\end{frame}

\begin{frame}{Símbolos Padronizados}
\textbf{Símbolos Padronizados:}
\begin{itemize}
\item \textbf{Sistema Central}: Círculo no centro
\item \textbf{Usuários}: Retângulos verdes
\item \textbf{Sistemas Externos}: Retângulos laranja
\item \textbf{Dispositivos}: Círculos vermelhos
\end{itemize}

\vspace{0.3cm}
\textbf{Tipos de Fluxos:}
\begin{itemize}
\item \textbf{Unidirecional}: $\rightarrow$ (dados, comandos)
\item \textbf{Bidirecional}: $\leftrightarrow$ (consultas, interações)
\item \textbf{Rótulos}: Descrevem o conteúdo do fluxo
\end{itemize}
\end{frame}



% Seção 5: Aplicação Prática
\section{Aplicação Prática: Sistema para ONG de Animais}

\begin{frame}{Cenário: Sistema de Gestão para ONG de Animais}
\textbf{Problema}: Uma ONG de proteção animal precisa de um sistema integrado para:
\begin{itemize}
\item Gerenciar animais resgatados
\item Controlar adoções
\item Administrar voluntários
\item Organizar eventos de arrecadação
\item Manter histórico veterinário
\end{itemize}

\vspace{0.5cm}
\textbf{Funcionalidades Necessárias:}
\begin{itemize}
\item Gerenciar animais resgatados
\item Controlar adoções
\item Administrar voluntários
\item Organizar eventos de arrecadação
\item Manter histórico veterinário
\end{itemize}
\end{frame}

\begin{frame}{Identificação de Atores Externos - Usuários}
\textbf{Atividade}: Em grupos de 3-4 pessoas, identifiquem os atores externos:

\textbf{Usuários Humanos:}
\begin{itemize}
\item Administradores da ONG
\item Voluntários
\item Adotantes
\item Veterinários parceiros
\end{itemize}

\vspace{0.5cm}
\textbf{Sistemas Externos:}
\begin{itemize}
\item Sistema de pagamento online
\item Redes sociais
\item Sistema de email
\item Banco de dados veterinário
\end{itemize}
\end{frame}

\begin{frame}{Identificação de Atores Externos - Dispositivos}
\textbf{Dispositivos:}
\begin{itemize}
\item Scanner de microchip
\item Câmera para fotos
\item Impressora de etiquetas
\end{itemize}

\vspace{0.5cm}
\textbf{Objetivo da Atividade:}
\begin{itemize}
\item Identificar todas as entidades externas
\item Definir interfaces necessárias
\item Compreender o contexto do sistema
\end{itemize}
\end{frame}

\begin{frame}{Modelo de Contexto - Sistema ONG de Animais}
\textbf{Sistema Central:} Sistema de Gestão ONG de Animais

\textbf{Atores Externos:}
\begin{itemize}
\item \textbf{Usuários Humanos:} Administradores, Voluntários, Adotantes, Veterinários
\item \textbf{Sistemas Externos:} Sistema de Pagamento, Redes Sociais, Sistema de Email
\item \textbf{Dispositivos:} Scanner de Microchip
\end{itemize}
\end{frame}

\begin{frame}{Fluxos Principais - Sistema ONG de Animais}
\textbf{Fluxos Principais:}
\begin{itemize}
\item Gestão (Administradores $\leftrightarrow$ Sistema)
\item Atividades (Voluntários $\leftrightarrow$ Sistema)
\item Adoção (Adotantes $\leftrightarrow$ Sistema)
\item Pagamentos (Sistema $\leftrightarrow$ Sistema de Pagamento)
\end{itemize}

\vspace{0.3cm}
\textbf{Fluxos Adicionais:}
\begin{itemize}
\item Compartilhamento (Sistema $\leftrightarrow$ Redes Sociais)
\item Notificações (Sistema $\leftrightarrow$ Email)
\item Histórico Médico (Veterinários $\leftrightarrow$ Sistema)
\item Identificação (Scanner $\leftrightarrow$ Sistema)
\end{itemize}
\end{frame}

\begin{frame}{Fluxos de Dados - Usuários}
\textbf{Administradores $\leftrightarrow$ Sistema:}
\begin{itemize}
\item Cadastro de animais e voluntários
\item Relatórios de gestão
\item Configurações do sistema
\end{itemize}

\vspace{0.3cm}
\textbf{Voluntários $\leftrightarrow$ Sistema:}
\begin{itemize}
\item Cadastro de atividades
\item Upload de fotos dos animais
\item Comunicação com adotantes
\end{itemize}
\end{frame}

\begin{frame}{Fluxos de Dados - Sistemas Externos}
\textbf{Sistema $\leftrightarrow$ Sistema de Pagamento:}
\begin{itemize}
\item Doações online
\item Taxas de adoção
\item Relatórios financeiros
\end{itemize}

\vspace{0.3cm}
\textbf{Sistema $\leftrightarrow$ Redes Sociais:}
\begin{itemize}
\item Compartilhamento de animais para adoção
\item Eventos de arrecadação
\item Campanhas de conscientização
\end{itemize}
\end{frame}

% Seção 6: Exercícios de Sala de Aula
\section{Exercícios de Sala de Aula}

\begin{frame}{Exercício 1: Sistema de Adoção Online}
\textbf{Cenário}: Expandir o sistema da ONG para incluir adoção online

\textbf{Tarefa}: Em grupos de 3-4 pessoas, desenvolvam:
\begin{enumerate}
\item Modelo de contexto de alto nível
\item Identificação de novos atores externos
\item Fluxos de dados adicionais
\end{enumerate}

\vspace{0.5cm}
\textbf{Tempo}: 15 minutos
\end{frame}

\begin{frame}{Exercício 1: Apresentação e Discussão}
\textbf{Apresentação}: Cada grupo apresenta seu modelo

\vspace{0.5cm}
\textbf{Discussão}: Comparar diferentes abordagens

\vspace{0.5cm}
\textbf{Pontos de Discussão:}
\begin{itemize}
\item Diferenças nos modelos propostos
\item Atores externos identificados
\item Complexidade dos fluxos
\end{itemize}
\end{frame}

\begin{frame}{Exercício 2: Sistema de Monitoramento de Animais}
\textbf{Cenário}: Sistema para monitorar animais em tratamento

\textbf{Tarefa}: Individualmente, criem um modelo de contexto considerando:
\begin{enumerate}
\item Sensores de temperatura e movimento
\item Veterinários remotos
\item Sistema de alertas
\item Banco de dados médico
\end{enumerate}

\vspace{0.5cm}
\textbf{Tempo}: 10 minutos
\end{frame}

\begin{frame}{Exercício 2: Validação e Refinamento}
\textbf{Validação}: Trocar com colega e validar completude

\vspace{0.5cm}
\textbf{Refinamento}: Adicionar fluxos de dados detalhados

\vspace{0.5cm}
\textbf{Pontos de Validação:}
\begin{itemize}
\item Todos os atores externos identificados
\item Fluxos de dados completos
\item Interfaces bem definidas
\end{itemize}
\end{frame}

\begin{frame}{Exercício 3: Sistema de Eventos e Arrecadação}
\textbf{Cenário}: Sistema para organizar eventos de arrecadação da ONG

\textbf{Tarefa}: Em duplas, desenvolvam:
\begin{enumerate}
\item Modelo de contexto de alto nível
\item Modelo detalhado com protocolos
\item Lista de requisitos derivados
\end{enumerate}

\vspace{0.5cm}
\textbf{Tempo}: 20 minutos
\end{frame}

\begin{frame}{Exercício 3: Apresentação e Avaliação}
\textbf{Apresentação}: Cada dupla apresenta ambos os modelos

\vspace{0.5cm}
\textbf{Avaliação}: Critérios: completude, clareza, consistência

\vspace{0.5cm}
\textbf{Pontos de Avaliação:}
\begin{itemize}
\item Identificação completa de atores externos
\item Fluxos de dados bem definidos
\item Consistência entre modelos
\end{itemize}
\end{frame}

% Seção 7: Casos de Estudo
\section{Casos de Estudo}

\begin{frame}{Caso 1: Sistema de Telemedicina Veterinária}
\textbf{Desafio}: Modelar contexto de plataforma de consultas veterinárias online

\textbf{Complexidades}:
\begin{itemize}
\item Múltiplos tipos de profissionais veterinários
\item Integração com sistemas hospitalares
\item Conformidade com regulamentações veterinárias
\item Qualidade de serviço (latência, disponibilidade)
\end{itemize}

\vspace{0.5cm}
\textbf{Modelo de Contexto:}
\begin{itemize}
\item \textbf{Sistema Central}: Sistema de Telemedicina Veterinária
\item \textbf{Atores}: Veterinários, Tutores, Sistema Hospitalar, Sistema de Pagamento
\item \textbf{Fluxos}: Consultas remotas, Histórico médico, Pagamentos online
\end{itemize}
\end{frame}

\begin{frame}{Caso 1: Lições Aprendidas}
\textbf{Lições Aprendidas}:
\begin{itemize}
\item Importância de requisitos não funcionais
\item Necessidade de validação legal
\item Consideração de aspectos éticos
\end{itemize}

\vspace{0.5cm}
\textbf{Aspectos Críticos}:
\begin{itemize}
\item Segurança de dados médicos
\item Confiabilidade do sistema
\item Experiência do usuário
\end{itemize}
\end{frame}

\begin{frame}{Caso 2: Sistema de IoT para Monitoramento Animal}
\textbf{Desafio}: Modelar contexto de sistema de monitoramento animal com IoT

\textbf{Complexidades}:
\begin{itemize}
\item Múltiplos tipos de sensores (temperatura, GPS, atividade)
\item Conectividade em áreas rurais
\item Análise de dados em tempo real
\item Integração com sistemas de alerta
\end{itemize}

\vspace{0.5cm}
\textbf{Modelo de Contexto:}
\begin{itemize}
\item \textbf{Sistema Central}: Sistema IoT de Monitoramento
\item \textbf{Sensores}: Temperatura, GPS, Atividade
\item \textbf{Sistemas}: Sistema de Alerta, Banco de Dados
\item \textbf{Fluxos}: Dados dos sensores → Análise → Alertas
\end{itemize}
\end{frame}

\begin{frame}{Caso 2: Lições Aprendidas}
\textbf{Lições Aprendidas}:
\begin{itemize}
\item Importância da robustez em ambientes adversos
\item Necessidade de processamento distribuído
\item Consideração de escalabilidade
\end{itemize}

\vspace{0.5cm}
\textbf{Aspectos Técnicos}:
\begin{itemize}
\item Confiabilidade da conectividade
\item Processamento de dados em tempo real
\item Interface com sistemas externos
\end{itemize}
\end{frame}

\begin{frame}{Caso 3: Sistema de Adoção com IA}
\textbf{Desafio}: Modelar contexto de sistema de adoção com inteligência artificial

\textbf{Complexidades}:
\begin{itemize}
\item Algoritmos de matching entre animais e adotantes
\item Integração com redes sociais
\item Sistema de recomendação
\item Análise de comportamento
\end{itemize}

\vspace{0.5cm}
\textbf{Modelo de Contexto:}
\begin{itemize}
\item \textbf{Sistema Central}: Sistema IA de Adoção
\item \textbf{Componentes}: Algoritmo Matching, Análise de Comportamento
\item \textbf{Sistemas}: Redes Sociais, Sistema de Recomendação
\item \textbf{Fluxos}: Matching inteligente → Recomendações → Compartilhamento
\end{itemize}
\end{frame}

\begin{frame}{Caso 3: Lições Aprendidas}
\textbf{Lições Aprendidas}:
\begin{itemize}
\item Importância da transparência em algoritmos
\item Necessidade de validação ética
\item Consideração de privacidade de dados
\end{itemize}

\vspace{0.5cm}
\textbf{Aspectos Éticos}:
\begin{itemize}
\item Transparência algorítmica
\item Proteção de dados pessoais
\item Validação de recomendações
\end{itemize}
\end{frame}

% Seção 8: Próximos Passos
\section{Próximos Passos}

\begin{frame}{Modelos de Dados}
\textbf{Modelos de Dados:}
\begin{itemize}
\item Entidade-Relacionamento
\item Modelagem conceitual
\item Normalização
\end{itemize}

\vspace{0.5cm}
\textbf{Aplicações:}
\begin{itemize}
\item Design de bancos de dados
\item Estruturação de informações
\item Integridade de dados
\end{itemize}
\end{frame}

\begin{frame}{Modelos de Processo}
\textbf{Modelos de Processo:}
\begin{itemize}
\item Diagramas de fluxo de dados
\item Diagramas de atividade
\item Workflows
\end{itemize}

\vspace{0.5cm}
\textbf{Aplicações:}
\begin{itemize}
\item Mapeamento de processos
\item Otimização de fluxos
\item Automação de tarefas
\end{itemize}
\end{frame}

\begin{frame}{Modelos de Arquitetura}
\textbf{Modelos de Arquitetura:}
\begin{itemize}
\item Arquitetura de software
\item Padrões arquiteturais
\item Microserviços
\end{itemize}

\vspace{0.5cm}
\textbf{Aplicações:}
\begin{itemize}
\item Design de sistemas
\item Escalabilidade
\item Manutenibilidade
\end{itemize}
\end{frame}

\begin{frame}{Leitura Recomendada}
\textbf{Livros Fundamentais:}
\begin{itemize}
\item \textbf{Sommerville, I.} Engenharia de Software - Capítulo 5
\item \textbf{Pressman, R.} Engenharia de Software - Capítulo 8
\end{itemize}

\vspace{0.5cm}
\textbf{Referências Adicionais:}
\begin{itemize}
\item \textbf{Boehm, B.} Software Engineering Economics
\item \textbf{ISO/IEC 25010} - Quality Model for Software Product
\end{itemize}
\end{frame}

\begin{frame}{Atividades Práticas}
\textbf{Projeto Prático}: Escolher um sistema real e desenvolver modelo de contexto completo

\vspace{0.5cm}
\textbf{Análise Comparativa}: Comparar diferentes ferramentas de modelagem

\vspace{0.5cm}
\textbf{Apresentação}: Apresentar modelo desenvolvido e justificar decisões

\vspace{0.5cm}
\textbf{Entrega}: Modelo de contexto + relatório de análise
\end{frame}

% Slide final
\begin{frame}{Dúvidas e Discussão}
\begin{center}
\textbf{Obrigado pela atenção!}
\end{center}

\textbf{Pontos-Chave da Aula}:
\begin{enumerate}
\item \textbf{Modelos de Contexto} são fundamentais para compreender o ambiente do sistema
\item \textbf{Atores externos} e suas interações devem ser identificados claramente
\item \textbf{Diagramas de contexto} facilitam comunicação e análise
\item \textbf{Aplicação prática} é essencial para consolidação do aprendizado
\end{enumerate}

\vspace{0.5cm}
\begin{center}
\includegraphics[width=0.4\textwidth]{../../Image/ufla-logo.PNG}
\end{center}

\vspace{0.3cm}
\textbf{Próxima Aula}: Modelos de Dados e Entidade-Relacionamento

\vspace{0.3cm}
\textbf{Contato}: [email do professor]\\
\textbf{Material}: Disponível no sistema acadêmico
\end{frame}

\end{document}
\documentclass[aspectratio=169]{beamer}

% Configurações do tema
\usetheme{default}
\usecolortheme{spruce}
\setbeamertemplate{navigation symbols}{}
\setbeamertemplate{footline}[frame number]

% Configurações para evitar overflow
\setbeamersize{text margin left=0.5cm}
\setbeamersize{text margin right=0.5cm}
\setbeamertemplate{itemize items}[circle]
\setbeamertemplate{enumerate items}[default]

% Configurações de fonte para evitar overflow
\setbeamerfont{itemize item}{size=\footnotesize}
\setbeamerfont{itemize subitem}{size=\tiny}
\setbeamerfont{enumerate item}{size=\footnotesize}
\setbeamerfont{section in toc}{size=\tiny}
\setbeamerfont{subsection in toc}{size=\tiny}

% Configurações específicas para o sumário
\setbeamertemplate{section in toc}[sections numbered]
\setbeamertemplate{subsection in toc}[subsections numbered]
\setbeamertemplate{navigation symbols}{}

% Configuração do logo da UFLA (compacto)
\setbeamertemplate{headline}{%
  \begin{beamercolorbox}[wd=\paperwidth,ht=0.8cm,dp=0.1cm]{section in head/foot}%
    \begin{center}
      \vspace{0.05cm}
      \includegraphics[height=0.7cm]{../../Image/ufla-logo.PNG}
    \end{center}
  \end{beamercolorbox}%
}

% Configuração do título centralizado e compacto
\setbeamertemplate{frametitle}{%
  \begin{beamercolorbox}[wd=\paperwidth,ht=0.8cm,dp=0.2cm]{frametitle}%
    \begin{center}
      \insertframetitle
    \end{center}
  \end{beamercolorbox}%
}

% Pacotes necessários para caracteres especiais - UTF-8
\usepackage[utf8]{inputenc}
\usepackage[T1]{fontenc}
\usepackage[brazilian]{babel}
\usepackage{amsmath}
\usepackage{amsfonts}
\usepackage{amssymb}
\usepackage{graphicx}
\usepackage{hyperref}
\usepackage{multicol}
\usepackage{tikz}

% Configurações de cores
\definecolor{azul}{RGB}{0,51,102}
\definecolor{azulclaro}{RGB}{51,102,153}
\definecolor{verde}{RGB}{34,139,34}
\definecolor{laranja}{RGB}{255,140,0}
\setbeamercolor{title}{fg=white,bg=azul}
\setbeamercolor{frametitle}{fg=white,bg=azulclaro}
\setbeamercolor{section title}{fg=white,bg=azul}

% Informações do documento
\title{Projeto de Software}
\subtitle{GCT088 - Aula 1.2 - Conceitos de Orientação a Objetos}
\author{Bento Rafael Siqueira}
\institute{Universidade Federal de Lavras (UFLA)}
\date{\today}

\begin{document}

% Slide de título
\begin{frame}
\titlepage
\end{frame}

% Sumário da apresentação
\begin{frame}{Sumário}
\textbf{Conceitos de Orientação a Objetos:}
\begin{itemize}
    \item Abstração
    \item Classe e Relacionamentos
    \item Objetos e Construtores
    \item Encapsulamento
    \item Sobrecarga
    \item Herança
    \item Polimorfismo
    \item Atividade Integrada
\end{itemize}
\end{frame}

% Seção 1: Abstração
\section{Abstração}

\begin{frame}{O que é Abstração?}
\textbf{Definição:}
\begin{itemize}
    \item \textbf{Abstração} é o processo de simplificar sistemas complexos
    \item Foca nos aspectos essenciais, ignorando detalhes irrelevantes
    \item Permite representar conceitos do mundo real de forma simplificada
    \item Base para todos os outros conceitos de POO
\end{itemize}

\textbf{Exemplo Prático:}
\begin{itemize}
    \item \textbf{Carro:} Motor, rodas, volante, freios (essencial)
    \item \textbf{Ignora:} Composição química do plástico, detalhes do motor
    \item \textbf{Resultado:} Modelo simplificado e funcional
\end{itemize}
\end{frame}

\begin{frame}{Exemplo Prático: Sistema de Viagens}
\textbf{Abstração de um Destino:}
\begin{itemize}
    \item \textbf{Incluído:} Nome, país, cidade, clima, disponibilidade
    \item \textbf{Omitido:} Coordenadas geográficas, população, história detalhada
    \item \textbf{Foco:} Funcionalidades essenciais para o sistema
\end{itemize}

\textbf{Arquivo de Exemplo:}
\begin{itemize}
    \item \texttt{Exemplos/Abstracao.cs}
    \item Classe \texttt{Destino} com propriedades essenciais
    \item Métodos para reserva e verificação de disponibilidade
\end{itemize}
\end{frame}

% Seção 2: Classe (Associação, Composição e Agregação)
\section{Classe e Relacionamentos}

\begin{frame}{O que é uma Classe?}
\textbf{Definição:}
\begin{itemize}
    \item \textbf{Classe} é um modelo/template para criar objetos
    \item Define atributos (propriedades) e comportamentos (métodos)
    \item Representa um conceito abstrato do mundo real
    \item Serve como "planta" para instanciar objetos
\end{itemize}

\textbf{Componentes:}
\begin{itemize}
    \item \textbf{Atributos:} Características do objeto
    \item \textbf{Métodos:} Comportamentos do objeto
    \item \textbf{Construtores:} Inicialização do objeto
    \item \textbf{Relacionamentos:} Como se conecta com outras classes
\end{itemize}
\end{frame}

\begin{frame}{Relacionamentos entre Classes}
\textbf{Três tipos principais:}

\textbf{1. Associação:}
\begin{itemize}
    \item Relacionamento simples entre classes
    \item Uma classe "usa" ou "conhece" outra
    \item Independência entre os objetos
\end{itemize}

\textbf{2. Composição:}
\begin{itemize}
    \item Relacionamento "parte-de" forte
    \item Objeto parte não existe sem o todo
    \item Ciclo de vida dependente
\end{itemize}

\textbf{3. Agregação:}
\begin{itemize}
    \item Relacionamento "parte-de" fraco
    \item Objeto parte pode existir independentemente
    \item Ciclo de vida independente
\end{itemize}
\end{frame}

\begin{frame}{Exemplo Prático: Sistema de Agência de Viagens}
\textbf{Arquivo de Exemplo:}
\begin{itemize}
    \item \texttt{Exemplos/Relacionamentos.cs}
    \item \textbf{Associação:} Viajante "usa" Destino
    \item \textbf{Composição:} Agência "contém" Departamentos
    \item \textbf{Agregação:} Departamento "tem" Funcionários
\end{itemize}

\textbf{Características:}
\begin{itemize}
    \item Viajante pode existir sem destino específico
    \item Departamento não existe sem Agência
    \item Funcionário pode existir sem Departamento
\end{itemize}
\end{frame}

% Seção 3: Objetos e Construtores
\section{Objetos e Construtores}

\begin{frame}{O que é um Objeto?}
\textbf{Definição:}
\begin{itemize}
    \item \textbf{Objeto} é uma instância de uma classe
    \item Representa um exemplo específico do conceito
    \item Possui estado (valores dos atributos) e comportamento
    \item Existe em memória durante a execução do programa
\end{itemize}

\textbf{Características:}
\begin{itemize}
    \item \textbf{Identidade:} Cada objeto é único
    \item \textbf{Estado:} Valores atuais dos atributos
    \item \textbf{Comportamento:} Métodos que pode executar
    \item \textbf{Ciclo de vida:} Criação, uso e destruição
\end{itemize}
\end{frame}

\begin{frame}{Construtores}
\textbf{Definição:}
\begin{itemize}
    \item \textbf{Construtor} é um método especial para inicializar objetos
    \item Chamado automaticamente quando o objeto é criado
    \item Mesmo nome da classe
    \item Pode ter parâmetros para configurar o estado inicial
\end{itemize}

\textbf{Tipos de Construtores:}
\begin{itemize}
    \item \textbf{Construtor Padrão:} Sem parâmetros
    \item \textbf{Construtor Parametrizado:} Com parâmetros
    \item \textbf{Construtor de Cópia:} Copia outro objeto
\end{itemize}
\end{frame}

\begin{frame}{Exemplo Prático: Sistema de Pacotes de Viagem}
\textbf{Arquivo de Exemplo:}
\begin{itemize}
    \item \texttt{Exemplos/ObjetosConstrutores.cs}
    \item Classe \texttt{PacoteViagem} com múltiplos construtores
    \item Demonstração de sobrecarga de construtores
\end{itemize}

\textbf{Construtores Implementados:}
\begin{itemize}
    \item \textbf{Padrão:} Valores padrão para todos os atributos
    \item \textbf{Parametrizado:} Código e nome do pacote
    \item \textbf{Completo:} Todos os atributos configuráveis
\end{itemize}
\end{frame}

% Seção 4: Encapsulamento
\section{Encapsulamento}

\begin{frame}{O que é Encapsulamento?}
\textbf{Definição:}
\begin{itemize}
    \item \textbf{Encapsulamento} é o princípio de ocultar detalhes internos
    \item Agrupa dados e métodos que operam sobre esses dados
    \item Controla o acesso aos dados através de interfaces
    \item Protege a integridade dos dados
\end{itemize}

\textbf{Benefícios:}
\begin{itemize}
    \item \textbf{Segurança:} Dados protegidos contra acesso indevido
    \item \textbf{Flexibilidade:} Implementação pode mudar sem afetar usuários
    \item \textbf{Manutenibilidade:} Mudanças localizadas
    \item \textbf{Reutilização:} Código mais modular
\end{itemize}
\end{frame}

\begin{frame}{Exemplo Prático: Controle de Reservas}
\textbf{Arquivo de Exemplo:}
\begin{itemize}
    \item \texttt{Exemplos/Encapsulamento.cs}
    \item Classe \texttt{ReservaViagem} com encapsulamento
    \item Atributos privados com controle de acesso
\end{itemize}

\textbf{Características:}
\begin{itemize}
    \item \textbf{Atributos privados:} valorTotal, senhaAcesso, confirmada
    \item \textbf{Propriedades públicas:} Controle de leitura/escrita
    \item \textbf{Métodos públicos:} Operações seguras
    \item \textbf{Validação:} Verificação de senha para operações críticas
\end{itemize}
\end{frame}

% Seção 5: Sobrecarga de Métodos
\section{Sobrecarga de Métodos}

\begin{frame}{O que é Sobrecarga de Métodos?}
\textbf{Definição:}
\begin{itemize}
    \item \textbf{Sobrecarga} permite múltiplas versões do mesmo método
    \item Métodos com mesmo nome mas parâmetros diferentes
    \item Compilador escolhe a versão adequada baseado nos argumentos
    \item Aumenta a flexibilidade e legibilidade do código
\end{itemize}

\textbf{Regras:}
\begin{itemize}
    \item \textbf{Nome igual} para todos os métodos
    \item \textbf{Parâmetros diferentes} (tipo, quantidade, ordem)
    \item \textbf{Retorno pode ser igual ou diferente}
    \item \textbf{Modificadores de acesso} podem variar
\end{itemize}
\end{frame}

\begin{frame}{Exemplo Prático: Calculadora de Viagens}
\textbf{Arquivo de Exemplo:}
\begin{itemize}
    \item \texttt{Exemplos/Sobrecarga.cs}
    \item Classe \texttt{CalculadoraViagem} com sobrecarga
    \item Múltiplas versões de métodos de cálculo
\end{itemize}

\textbf{Métodos Sobrecarregados:}
\begin{itemize}
    \item \textbf{CalcularPreco:} Diferentes combinações de parâmetros
    \item \textbf{AplicarDesconto:} Percentual ou tipo de cliente
    \item \textbf{ValidarDestino:} Validação simples ou com lista
\end{itemize}
\end{frame}

% Seção 6: Herança
\section{Herança}

\begin{frame}{O que é Herança?}
\textbf{Definição:}
\begin{itemize}
    \item \textbf{Herança} permite criar classes baseadas em outras
    \item Classe filha herda atributos e métodos da classe pai
    \item Promove reutilização de código
    \item Estabelece relacionamento "é-um" entre classes
\end{itemize}

\textbf{Benefícios:}
\begin{itemize}
    \item \textbf{Reutilização:} Código da classe pai é reutilizado
    \item \textbf{Extensibilidade:} Novas funcionalidades podem ser adicionadas
    \item \textbf{Hierarquia:} Organização lógica das classes
    \item \textbf{Polimorfismo:} Base para polimorfismo
\end{itemize}
\end{frame}

\begin{frame}{Exemplo Prático: Hierarquia de Transportes}
\textbf{Arquivo de Exemplo:}
\begin{itemize}
    \item \texttt{Exemplos/Heranca.cs}
    \item Classe base \texttt{Transporte}
    \item Classes filhas: \texttt{Aviao}, \texttt{Onibus}, \texttt{AviaoExecutivo}
\end{itemize}

\textbf{Hierarquia Implementada:}
\begin{itemize}
    \item \textbf{Transporte:} Classe base com propriedades comuns
    \item \textbf{Aviao:} Herança simples com propriedades específicas
    \item \textbf{Onibus:} Herança simples com características próprias
    \item \textbf{AviaoExecutivo:} Herança múltipla de níveis
\end{itemize}
\end{frame}

% Seção 7: Polimorfismo
\section{Polimorfismo}

\begin{frame}{O que é Polimorfismo?}
\textbf{Definição:}
\begin{itemize}
    \item \textbf{Polimorfismo} significa "muitas formas"
    \item Permite que objetos de classes diferentes respondam ao mesmo método
    \item Método pode ter implementações diferentes em classes diferentes
    \item Aumenta flexibilidade e extensibilidade do código
\end{itemize}

\textbf{Tipos:}
\begin{itemize}
    \item \textbf{Polimorfismo de Sobrescrita:} Método virtual/override
    \item \textbf{Polimorfismo de Sobrecarga:} Múltiplas versões do método
    \item \textbf{Polimorfismo de Interface:} Implementação de interfaces
\end{itemize}
\end{frame}

\begin{frame}{Exemplo Prático: Sistema de Serviços de Viagem}
\textbf{Arquivo de Exemplo:}
\begin{itemize}
    \item \texttt{Exemplos/Polimorfismo.cs}
    \item Classe abstrata \texttt{ServicoViagem}
    \item Classes filhas: \texttt{Hospedagem}, \texttt{Passeio}, \texttt{Transfer}
\end{itemize}

\textbf{Polimorfismo Demonstrado:}
\begin{itemize}
    \item \textbf{Métodos abstratos:} CalcularPrecoFinal(), ObterTipo()
    \item \textbf{Implementações específicas:} Cada classe filha implementa diferentemente
    \item \textbf{Uso polimórfico:} Lista de ServicoViagem com objetos diferentes
\end{itemize}
\end{frame}

% Seção 8: Atividade Prática Integrada
\section{Atividade Prática Integrada}

\begin{frame}{Sistema de Gestão de Viagens}
\textbf{Desafio:} Implementar um sistema completo de agência de viagens usando todos os conceitos aprendidos.

\textbf{Requisitos:}
\begin{itemize}
    \item \textbf{Abstração:} Modelar destinos, clientes, reservas
    \item \textbf{Classes e Relacionamentos:} Associação, composição, agregação
    \item \textbf{Objetos e Construtores:} Criar instâncias com diferentes construtores
    \item \textbf{Encapsulamento:} Proteger dados sensíveis
    \item \textbf{Sobrecarga:} Múltiplas formas de criar objetos
    \item \textbf{Herança:} Hierarquia de serviços e clientes
    \item \textbf{Polimorfismo:} Diferentes tipos de serviços
\end{itemize}
\end{frame}

\begin{frame}{Arquivo de Exemplo Integrado}
\textbf{Arquivo Principal:}
\begin{itemize}
    \item \texttt{Exemplos/SistemaViagem.cs}
    \item Sistema completo demonstrando todos os conceitos
    \item Classe \texttt{SistemaViagem} com método \texttt{DemonstrarSistema()}
\end{itemize}

\textbf{Conceitos Aplicados:}
\begin{itemize}
    \item \textbf{Abstração:} ServicoBase como conceito abstrato
    \item \textbf{Herança:} Hospedagem e Transporte herdam de ServicoBase
    \item \textbf{Polimorfismo:} ExibirDetalhes() e ObterTipo() implementados diferentemente
    \item \textbf{Relacionamentos:} Associação, composição e agregação
    \item \textbf{Encapsulamento:} Atributos protegidos e controle de acesso
\end{itemize}
\end{frame}

\begin{frame}{Conceitos Aplicados na Atividade}
\textbf{Verificação dos Conceitos:}

\textbf{$\checkmark$ Abstração:}
\begin{itemize}
    \item ServicoBase como conceito abstrato
    \item Foco nas propriedades essenciais
\end{itemize}

\textbf{$\checkmark$ Classe e Relacionamentos:}
\begin{itemize}
    \item Associação: Cliente usa Agência
    \item Composição: Agência contém Departamentos
    \item Agregação: Agência tem Serviços
\end{itemize}

\textbf{$\checkmark$ Objetos e Construtores:}
\begin{itemize}
    \item Múltiplos construtores para Hospedagem
    \item Inicialização flexível de objetos
\end{itemize}

\textbf{$\checkmark$ Encapsulamento:}
\begin{itemize}
    \item Atributos protegidos em ServicoBase
    \item Controle de acesso aos dados
\end{itemize}
\end{frame}

\begin{frame}{Conceitos Aplicados na Atividade (Continuação)}
\textbf{$\checkmark$ Sobrecarga de Métodos:}
\begin{itemize}
    \item Múltiplos construtores em Hospedagem
    \item Diferentes formas de criar objetos
\end{itemize}

\textbf{$\checkmark$ Herança:}
\begin{itemize}
    \item Hospedagem e Transporte herdam de ServicoBase
    \item Reutilização de código comum
\end{itemize}

\textbf{$\checkmark$ Polimorfismo:}
\begin{itemize}
    \item ExibirDetalhes() com implementações diferentes
    \item ObterTipo() retorna tipos específicos
    \item Lista de ServicoBase contém objetos diferentes
\end{itemize}

\textbf{Resultado:} Sistema completo e funcional demonstrando todos os conceitos!
\end{frame}

% Seção 9: Resumo
\section{Resumo}

\begin{frame}{Resumo dos Conceitos}
\textbf{Conceitos Fundamentais de POO:}

\textbf{1. Abstração:} Simplificar sistemas complexos
\textbf{2. Classe:} Modelo para criar objetos
\textbf{3. Objetos:} Instâncias com estado e comportamento
\textbf{4. Encapsulamento:} Proteção e controle de acesso
\textbf{5. Sobrecarga:} Múltiplas versões do mesmo método
\textbf{6. Herança:} Reutilização e extensão de código
\textbf{7. Polimorfismo:} Flexibilidade e extensibilidade

\textbf{Benefícios:}
\begin{itemize}
    \item Código mais organizado e reutilizável
    \item Manutenção mais fácil
    \item Extensibilidade aumentada
    \item Modelagem mais próxima do mundo real
\end{itemize}
\end{frame}

\begin{frame}{Próximos Passos}
\textbf{Análise e Projeto Orientado a Objetos:}
\begin{itemize}
    \item \textbf{Modelagem de Domínio:} Identificar entidades e relacionamentos
    \item \textbf{Diagramas UML:} Casos de uso, classes, sequência
    \item \textbf{Padrões de Análise:} GRASP e outros padrões
    \item \textbf{Refatoração:} Melhorar design de código existente
    \item \textbf{Padrões de Design:} GoF e outros padrões
    \item \textbf{Princípios SOLID:} Boas práticas de design
    \item \textbf{Arquitetura de Software:} Organização de sistemas
\end{itemize}

\textbf{Prática Recomendada:}
\begin{itemize}
    \item Modelar o sistema de viagens com UML
    \item Aplicar padrões de design
    \item Refatorar código existente
    \item Implementar arquiteturas escaláveis
\end{itemize}
\end{frame}

\begin{frame}{Obrigado!}
\begin{center}
\Large \textbf{Dúvidas?}
\vspace{1cm}
\normalsize
\textbf{Próxima aula:} Análise e Projeto Orientado a Objetos
\end{center}

\textbf{Contato:}
\begin{itemize}
    \item \textbf{Professor:} Bento Rafael Siqueira
    \item \textbf{Disciplina:} GCT088 - Projeto de Software
    \item \textbf{Universidade:} UFLA
\end{itemize}
\end{frame}

\end{document}

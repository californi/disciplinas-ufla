\documentclass[aspectratio=169]{beamer}

% Configurações do tema
\usetheme{Berkeley}
\usecolortheme{spruce}
\setbeamertemplate{navigation symbols}{}
\setbeamertemplate{footline}[frame number]

% Configuração do logo da UFLA
\setbeamertemplate{headline}{%
  \begin{beamercolorbox}[wd=\paperwidth,ht=1.2cm,dp=0.2cm]{section in head/foot}%
    \begin{center}
      \vspace{0.1cm}
      \includegraphics[height=1.0cm]{../../Image/ufla-logo.PNG}
    \end{center}
  \end{beamercolorbox}%
}

% Configuração do título centralizado
\setbeamertemplate{frametitle}{%
  \begin{beamercolorbox}[wd=\paperwidth,ht=1.2cm,dp=0.3cm]{frametitle}%
    \begin{center}
      \insertframetitle
    \end{center}
  \end{beamercolorbox}%
}

% Pacotes necessários
\usepackage[utf8]{inputenc}
\usepackage[brazilian]{babel}
\usepackage{amsmath}
\usepackage{amsfonts}
\usepackage{amssymb}
\usepackage{graphicx}
\usepackage{hyperref}
\usepackage{multicol}
\usepackage{tikz}
\usepackage{xcolor}

% Configurações de cores
\definecolor{azul}{RGB}{0,51,102}
\definecolor{azulclaro}{RGB}{51,102,153}
\definecolor{lightblue}{RGB}{173,216,230}
\definecolor{lightgreen}{RGB}{144,238,144}
\definecolor{lightcoral}{RGB}{240,128,128}
\definecolor{lightyellow}{RGB}{255,255,224}
\definecolor{lightpink}{RGB}{255,182,193}
\definecolor{lightgray}{RGB}{211,211,211}
\definecolor{orange}{RGB}{255,165,0}

\setbeamercolor{title}{fg=white,bg=azul}
\setbeamercolor{frametitle}{fg=white,bg=azulclaro}
\setbeamercolor{section title}{fg=white,bg=azul}

% Informações do documento
\title[Modelos de Interação]{Modelos de Interação}
\subtitle{GCT088 - Projeto de Software}
\author{Prof. Dr. Bento Rafael Siqueira}
\institute{Universidade Federal de Lavras\\Departamento de Ciência da Computação}
\date{\today}

\begin{document}

% Slide de título
\begin{frame}
\titlepage
\end{frame}

% Sumário
\begin{frame}{Sumário}
\tableofcontents
\end{frame}

% Seção 1: Introdução
\section{Introdução}

\begin{frame}{O que são Modelos de Interação?}
\textbf{Definição:}
\begin{itemize}
\item Representam como o sistema interage com usuários e sistemas externos
\item Descrevem sequências de ações e trocas de mensagens
\item Focam no comportamento dinâmico do sistema
\end{itemize}

\vspace{0.5cm}
\textbf{Objetivos:}
\begin{itemize}
\item Especificar interfaces do sistema
\item Identificar requisitos funcionais
\item Facilitar comunicação entre stakeholders
\end{itemize}
\end{frame}

\begin{frame}{Importância dos Modelos de Interação}
\textbf{Benefícios:}
\begin{itemize}
\item \textbf{Comunicação}: Facilitam entendimento entre desenvolvedores e usuários
\item \textbf{Validação}: Permitem validar requisitos antes da implementação
\item \textbf{Documentação}: Servem como base para especificação de interfaces
\end{itemize}

\vspace{0.5cm}
\textbf{Aplicações:}
\begin{itemize}
\item Análise de requisitos
\item Design de interfaces
\item Testes de sistema
\end{itemize}
\end{frame}

% Seção 2: Diagramas de Casos de Uso
\section{Diagramas de Casos de Uso}

\begin{frame}{Conceitos Fundamentais}
\textbf{Caso de Uso:}
\begin{itemize}
\item Descrição de uma funcionalidade do sistema
\item Interação entre ator e sistema
\item Representa um objetivo específico do usuário
\end{itemize}

\vspace{0.5cm}
\textbf{Ator:}
\begin{itemize}
\item Entidade externa que interage com o sistema
\item Pode ser humano, sistema ou dispositivo
\item Representa papel ou função
\end{itemize}
\end{frame}

\begin{frame}{Elementos dos Diagramas de Casos de Uso}
\textbf{Símbolos Principais:}
\begin{itemize}
\item \textbf{Caso de Uso}: Elipse com nome descritivo
\item \textbf{Ator}: Figura humana ou retângulo com nome
\item \textbf{Sistema}: Retângulo contendo os casos de uso
\item \textbf{Relacionamentos}: Linhas conectando elementos
\end{itemize}

\vspace{0.5cm}
\textbf{Tipos de Relacionamentos:}
\begin{itemize}
\item \textbf{Associação}: Ator executa caso de uso
\item \textbf{Inclusão}: Caso de uso sempre executado
\item \textbf{Extensão}: Caso de uso condicionalmente executado
\end{itemize}
\end{frame}

\begin{frame}{Exemplo: Sistema de "Biblioteca" de Bebidas}
\textbf{Atores:}
\begin{itemize}
\item "Bibliotecário" da República (o morador mais organizado)
\item "Usuário da Biblioteca" (estudante que precisa de uma "leitura")
\item Sistema de pagamento (para multas quando alguém "esquece" de devolver)
\end{itemize}

\vspace{0.5cm}
\textbf{Casos de Uso Principais:}
\begin{itemize}
\item "Emprestar" bebida (pegar uma gelada da geladeira)
\item "Devolver" bebida (comprar uma nova para repor)
\item Renovar "empréstimo" (pegar mais uma porque a primeira não foi suficiente)
\item Consultar disponibilidade (tem cerveja gelada?)
\end{itemize}
\end{frame}

% Seção 3: Diagramas de Sequência
\section{Diagramas de Sequência}

\begin{frame}{Conceitos dos Diagramas de Sequência}
\textbf{Definição:}
\begin{itemize}
\item Mostram interações entre objetos ao longo do tempo
\item Representam sequência de mensagens trocadas
\item Focam na ordem temporal das interações
\end{itemize}

\vspace{0.5cm}
\textbf{Elementos Principais:}
\begin{itemize}
\item \textbf{Objetos}: Participantes da interação
\item \textbf{Mensagens}: Comunicação entre objetos
\item \textbf{Linha de vida}: Representa existência do objeto
\item \textbf{Ativação}: Período de execução de método
\end{itemize}
\end{frame}

\begin{frame}{Tipos de Mensagens}
\textbf{Síncronas:}
\begin{itemize}
\item Chamada de método
\item Requer resposta antes de continuar
\item Representada por seta sólida com cabeça preenchida
\end{itemize}

\vspace{0.3cm}
\textbf{Assíncronas:}
\begin{itemize}
\item Mensagem enviada sem esperar resposta
\item Representada por seta sólida com cabeça aberta
\end{itemize}

\vspace{0.3cm}
\textbf{Retorno:}
\begin{itemize}
\item Resposta a uma mensagem síncrona
\item Representada por seta tracejada
\end{itemize}
\end{frame}

\begin{frame}{Exemplo: "Empréstimo" de Bebida}
\textbf{Sequência de Interações (A Saga da Cerveja Gelada):}
\begin{enumerate}
\item Morador solicita "empréstimo" (abre a geladeira com esperança)
\item Sistema verifica disponibilidade (tem cerveja gelada?)
\item Sistema registra "empréstimo" (anota quem pegou)
\item Sistema atualiza estoque (diminui uma cerveja)
\item Sistema confirma "empréstimo" (pode beber à vontade!)
\end{enumerate}

\vspace{0.5cm}
\textbf{Objetos Envolvidos:}
\begin{itemize}
\item Interface do usuário (celular do morador)
\item Controlador de "empréstimo" (morador mais organizado)
\item Banco de dados (memória do morador responsável)
\item Sistema de notificação (WhatsApp da república)
\end{itemize}
\end{frame}

% Seção 4: Boas Práticas
\section{Boas Práticas}

\begin{frame}{Boas Práticas - Casos de Uso}
\textbf{Nomenclatura:}
\begin{itemize}
\item Use verbos no infinitivo (Emprestar livro)
\item Seja específico e claro
\item Evite termos técnicos desnecessários
\end{itemize}

\vspace{0.5cm}
\textbf{Organização:}
\begin{itemize}
\item Agrupe casos relacionados
\item Use relacionamentos apropriados
\item Mantenha diagramas simples
\end{itemize}
\end{frame}

\begin{frame}{Boas Práticas - Diagramas de Sequência}
\textbf{Clareza:}
\begin{itemize}
\item Use nomes descritivos para objetos
\item Limite número de objetos por diagrama
\item Foque em um cenário específico
\end{itemize}

\vspace{0.5cm}
\textbf{Detalhamento:}
\begin{itemize}
\item Inclua condições e loops quando necessário
\item Use fragmentos combinados para complexidade
\item Documente exceções importantes
\end{itemize}
\end{frame}

% Seção 5: Aplicação Prática
\section{Aplicação Prática: Sistema de Gestão de Festas Universitárias}

\begin{frame}{Cenário: Sistema de Gestão de Festas Universitárias}
\textbf{Problema}: Desenvolver modelos de interação para um sistema de organização de festas universitárias

\textbf{Requisitos Principais:}
\begin{itemize}
\item Organização de eventos
\item Gestão de convidados
\item Controle de entrada
\item Venda de bebidas
\end{itemize}

\vspace{0.5cm}
\textbf{Atores Identificados:}
\begin{itemize}
\item Organizador de festa
\item Convidado
\item Segurança
\item Fornecedor de bebidas
\end{itemize}
\end{frame}

\begin{frame}{Atividade 1: Casos de Uso}
\textbf{Tarefa}: Em grupos de 3-4 pessoas, criem um diagrama de casos de uso para o sistema de gestão de festas universitárias

\textbf{Foque em:}
\begin{itemize}
\item Identificação de atores (organizadores, convidados, segurança)
\item Casos de uso principais (criar festa, convidar pessoas, vender bebidas)
\item Relacionamentos entre casos
\end{itemize}

\vspace{0.5cm}
\textbf{Tempo}: 20 minutos

\vspace{0.5cm}
\textbf{Apresentação}: Cada grupo apresenta seu diagrama
\end{frame}

\begin{frame}{Atividade 2: Diagrama de Sequência}
\textbf{Tarefa}: Individualmente, criem um diagrama de sequência para o processo de organização de uma festa

\textbf{Considere:}
\begin{itemize}
\item Criação do evento
\item Convite de pessoas
\item Controle de entrada
\item Venda de bebidas
\end{itemize}

\vspace{0.5cm}
\textbf{Tempo}: 25 minutos

\vspace{0.5cm}
\textbf{Discussão}: Compare diferentes abordagens
\end{frame}

% Seção 6: Exercícios de Sala de Aula
\section{Exercícios de Sala de Aula}

\begin{frame}{Exercício 1: Sistema de Delivery de Bebidas}
\textbf{Cenário}: Sistema de delivery de bebidas para festas universitárias

\textbf{Tarefa}: Em duplas, desenvolvam:
\begin{enumerate}
\item Diagrama de casos de uso
\item Diagrama de sequência para pedido de bebidas
\end{enumerate}

\vspace{0.5cm}
\textbf{Tempo}: 30 minutos

\vspace{0.5cm}
\textbf{Avaliação}: Completude e clareza dos diagramas
\end{frame}

\begin{frame}{Exercício 2: Sistema de Bar Universitário}
\textbf{Cenário}: Sistema de gestão de bar universitário

\textbf{Tarefa}: Individualmente, criem:
\begin{enumerate}
\item Casos de uso para diferentes tipos de usuários (clientes, garçons, gerente)
\item Sequência para processo de pedido e pagamento
\end{enumerate}

\vspace{0.5cm}
\textbf{Tempo}: 25 minutos

\vspace{0.5cm}
\textbf{Foco}: Diferenciação de papéis e fluxos
\end{frame}

\begin{frame}{Exercício 3: Sistema de República Universitária}
\textbf{Cenário}: Sistema de gestão de república universitária

\textbf{Tarefa}: Em grupos de 4 pessoas, desenvolvam:
\begin{enumerate}
\item Modelo completo de casos de uso
\item Diagramas de sequência para organização de festas e rateio de contas
\end{enumerate}

\vspace{0.5cm}
\textbf{Tempo}: 40 minutos

\vspace{0.5cm}
\textbf{Apresentação}: Demonstração dos modelos criados
\end{frame}

% Seção 7: Casos de Estudo
\section{Casos de Estudo}

\begin{frame}{Caso 1: Sistema de Gestão de Eventos Universitários}
\textbf{Desafio}: Modelar interações para sistema de eventos universitários

\textbf{Complexidades:}
\begin{itemize}
\item Múltiplos tipos de eventos (festas, shows, esportes)
\item Controle de idade e segurança
\item Integração com sistemas de pagamento
\end{itemize}

\vspace{0.5cm}
\textbf{Lições Aprendidas:}
\begin{itemize}
\item Importância do controle de idade
\item Necessidade de validações de segurança
\item Consideração de capacidade do local
\end{itemize}
\end{frame}

\begin{frame}{Caso 2: Sistema de Delivery de Bebidas Universitário}
\textbf{Desafio}: Modelar sistema de delivery de bebidas para estudantes

\textbf{Complexidades:}
\begin{itemize}
\item Verificação de idade e documentos
\item Integração com múltiplos fornecedores
\item Sistema de avaliação e confiança
\end{itemize}

\vspace{0.5cm}
\textbf{Aspectos Críticos:}
\begin{itemize}
\item Segurança na entrega
\item Controle de idade rigoroso
\item Experiência do usuário
\end{itemize}
\end{frame}

% Seção 8: Próximos Passos
\section{Próximos Passos}

\begin{frame}{Tópicos Relacionados}
\textbf{Modelos de Dados:}
\begin{itemize}
\item Diagramas de classe
\item Modelagem de entidades
\item Relacionamentos entre dados
\end{itemize}

\vspace{0.5cm}
\textbf{Modelos de Arquitetura:}
\begin{itemize}
\item Diagramas de componentes
\item Arquitetura de software
\item Padrões arquiteturais
\end{itemize}
\end{frame}

\begin{frame}{Leitura Recomendada}
\textbf{Livros Fundamentais:}
\begin{itemize}
\item \textbf{Sommerville, I.} Engenharia de Software - Capítulo 5.2
\item \textbf{Larman, C.} UML e Padrões - Capítulos 6-7
\end{itemize}

\vspace{0.5cm}
\textbf{Referências Adicionais:}
\begin{itemize}
\item \textbf{Cockburn, A.} Writing Effective Use Cases
\item \textbf{Booch, G.} Object-Oriented Analysis and Design
\end{itemize}
\end{frame}

\begin{frame}{Atividades Práticas}
\textbf{Projeto Prático}: Escolher um sistema real e desenvolver modelos de interação completos

\vspace{0.5cm}
\textbf{Ferramentas}: Utilizar ferramentas como StarUML, Lucidchart ou Draw.io

\vspace{0.5cm}
\textbf{Entrega}: Modelos de casos de uso e sequência + relatório de análise
\end{frame}

% Slide final
\begin{frame}{Dúvidas e Discussão}
\begin{center}
\textbf{Obrigado pela atenção!}

\vspace{1cm}
\textbf{Dúvidas?}
\end{center}
\end{frame}

\end{document}

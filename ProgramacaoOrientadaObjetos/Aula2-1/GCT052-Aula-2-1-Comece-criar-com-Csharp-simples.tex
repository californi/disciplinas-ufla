\documentclass[aspectratio=169]{beamer}

% Configuracoes do tema
\usetheme{Berkeley}
\usecolortheme{spruce}
\setbeamertemplate{navigation symbols}{}
\setbeamertemplate{footline}[frame number]

% Configuracoes para evitar overflow
\setbeamersize{text margin left=0.5cm}
\setbeamersize{text margin right=0.5cm}
\setbeamertemplate{itemize items}[circle]
\setbeamertemplate{enumerate items}[default]

% Configuracoes de fonte para evitar overflow
\setbeamerfont{itemize item}{size=\footnotesize}
\setbeamerfont{itemize subitem}{size=\tiny}
\setbeamerfont{enumerate item}{size=\footnotesize}

% Configuracoes especificas para o sumario
\setbeamertemplate{section in toc}[sections numbered]
\setbeamertemplate{subsection in toc}[subsections numbered]

% Configuracao do logo da UFLA (compacto)
\setbeamertemplate{headline}{%
  \begin{beamercolorbox}[wd=\paperwidth,ht=0.8cm,dp=0.1cm]{section in head/foot}%
    \begin{center}
      \vspace{0.05cm}
      \includegraphics[height=0.7cm]{../../Image/ufla-logo.PNG}
    \end{center}
  \end{beamercolorbox}%
}

% Configuracao do titulo centralizado e compacto
\setbeamertemplate{frametitle}{%
  \begin{beamercolorbox}[wd=\paperwidth,ht=0.8cm,dp=0.2cm]{frametitle}%
    \begin{center}
      \insertframetitle
    \end{center}
  \end{beamercolorbox}%
}

% Pacotes necessarios para caracteres especiais - UTF-8
\usepackage[utf8]{inputenc}
\usepackage[T1]{fontenc}
\usepackage[brazilian]{babel}
\usepackage{amsmath}
\usepackage{amsfonts}
\usepackage{amssymb}
\usepackage{graphicx}
\usepackage{hyperref}
\usepackage{multicol}
\usepackage{tikz}
% \usepackage{listings}

% Configuracoes de cores
\definecolor{azul}{RGB}{0,51,102}
\definecolor{azulclaro}{RGB}{51,102,153}
\definecolor{verde}{RGB}{34,139,34}
\definecolor{laranja}{RGB}{255,140,0}
\setbeamercolor{title}{fg=white,bg=azul}
\setbeamercolor{frametitle}{fg=white,bg=azulclaro}
\setbeamercolor{section title}{fg=white,bg=azul}

% Configuracoes do listings para codigo bonito e compacto
% \lstset{
%     basicstyle=\ttfamily\tiny,
%     breaklines=true,
%     frame=single,
%     numbers=none,
%     showstringspaces=false,
%     tabsize=2
% }

% Definicao da linguagem C#
% \lstdefinelanguage{CSharp}{
%     keywords={class, namespace, using, static, void, string, int, double, bool, if, else, switch, case, break, return, for, while, foreach, try, catch, finally, throw, new, this, base, public, private, protected, internal, abstract, virtual, override, sealed, interface, struct, enum, delegate, event, var, const, readonly, ref, out, params, in, async, await},
%     sensitive=false,
%     morecomment=[l]{//},
%     morestring=[b]"
% }

% Definicao da linguagem XAML
% \lstdefinelanguage{XML}{
%     keywords={xmlns, x, Window, Grid, StackPanel, Button, TextBox, TextBlock, Label, RowDefinition, ColumnDefinition, Margin, Height, Width, Content, Name, Class, Title},
%     sensitive=false,
%     morecomment=[s]{<!--}{-->},
%     morestring=[b]"
% }

% Informacoes do documento
\title{Programacao Orientada a Objetos}
\subtitle{GCT052 - Aula 2.1 - Comece a criar com C\#}
\author{Bento Rafael Siqueira}
\institute{Universidade Federal de Lavras (UFLA)}
\date{\today}

\begin{document}

% Slide de titulo
\begin{frame}
\titlepage
\end{frame}

% Slide de sumario
\begin{frame}{Sumario}
\tableofcontents
\end{frame}

% Secao 1: Introducao ao Capitulo
\section{Introducao ao Capitulo 1}

\begin{frame}{O que voce vai aprender hoje?}
\textbf{Objetivos da aula:}
\begin{itemize}
    \item \textbf{Objetivo:} Começar com Console e evoluir para WPF
    \item \textbf{Foco:} Aprendizado progressivo: Console → WPF
    \item \textbf{Resultado:} Base sólida + aplicação desktop funcional
    \item \textbf{Metodologia:} Aprender fazendo, do simples ao complexo
\end{itemize}

\textbf{Conceitos fundamentais:}
\begin{itemize}
    \item Estrutura básica de programas C\#
    \item Transição de aplicações Console para WPF
    \item Programação orientada a eventos
    \item Separação entre interface e lógica
\end{itemize}
\end{frame}

\begin{frame}{Por que comecar com Console?}
\textbf{Vantagens do Console para aprendizado:}
\begin{itemize}
    \item \textbf{Simplicidade:} Conceitos básicos sem distrações
    \item \textbf{Fundamentos:} Entrada/saída, variáveis, controle de fluxo
    \item \textbf{Progressão:} Console → WPF → Aplicações complexas
    \item \textbf{Base sólida:} Compreensão profunda dos conceitos
\end{itemize}

\textbf{Benefícios educacionais:}
\begin{itemize}
    \item Foco nos conceitos fundamentais
    \item Aprendizado sequencial e lógico
    \item Construção de base sólida
    \item Transição natural para interfaces gráficas
\end{itemize}
\end{frame}

\begin{frame}{Por que C\# e WPF?}
\textbf{Vantagens das tecnologias:}
\begin{itemize}
    \item \textbf{C\#:} Linguagem moderna, poderosa e versátil
    \item \textbf{WPF:} Framework avançado para interfaces gráficas
    \item \textbf{XAML:} Linguagem declarativa para UI
    \item \textbf{Integração:} C\# + WPF + XAML = Aplicações desktop robustas
    \item \textbf{Carreira:} Habilidades muito valorizadas no mercado
\end{itemize}

\textbf{Benefícios práticos:}
\begin{itemize}
    \item Desenvolvimento rápido e eficiente
    \item Interface rica e moderna
    \item Código organizado e manutenível
    \item Suporte oficial da Microsoft
\end{itemize}
\end{frame}

% Secao 2: Aplicacoes Console (Base)
\section{Aplicacoes Console (Base)}

\begin{frame}{O que e uma aplicacao Console?}
\textbf{Características principais:}
\begin{itemize}
    \item \textbf{Interface de linha de comando} simples
    \item \textbf{Entrada e saída} via texto
    \item \textbf{Execução sequencial} de comandos
    \item \textbf{Ideal para aprender} conceitos básicos
    \item \textbf{Base para scripts} e ferramentas
    \item \textbf{Portabilidade} entre sistemas operacionais
\end{itemize}

\textbf{Vantagens educacionais:}
\begin{itemize}
    \item Foco na lógica sem distrações visuais
    \item Entrada/saída simples e direta
    \item Execução linear e previsível
    \item Base para automação e scripts
\end{itemize}
\end{frame}

\begin{frame}{Estrutura basica de um programa Console}
\textbf{Componentes essenciais:}
\begin{itemize}
    \item \textbf{using System} - Importa bibliotecas básicas do .NET Framework
    \item \textbf{namespace} - Organiza o código em um espaço de nomes lógico
    \item \textbf{class Program} - Define a classe principal (base da POO)
    \item \textbf{static void Main} - Método principal, ponto de entrada da aplicação
\end{itemize}

\textbf{Conceitos fundamentais:}
\begin{itemize}
    \item Importação de bibliotecas - Acesso às funcionalidades do .NET
    \item Organização de código - Estruturação lógica e hierárquica
    \item Definição de classes - Base para criar objetos e estruturas
    \item Ponto de entrada - Onde a execução do programa começa
\end{itemize}
\end{frame}

\begin{frame}{Conceitos basicos demonstrados}
\textbf{Elementos fundamentais:}
\begin{itemize}
    \item \textbf{Importação de bibliotecas} - Acesso às funcionalidades do .NET
    \item \textbf{Organização de código} - Estruturação lógica e hierárquica
    \item \textbf{Definição de classes} - Base para criar objetos e estruturas
    \item \textbf{Ponto de entrada} - Onde a execução do programa começa
    \item \textbf{Entrada e saída} - Comunicação com o usuário
\end{itemize}

\textbf{Benefícios:}
\begin{itemize}
    \item Código organizado e estruturado
    \item Reutilização de funcionalidades
    \item Base sólida para POO
    \item Facilidade de manutenção
\end{itemize}
\end{frame}

\begin{frame}{Exemplo pratico: Calculadora simples}
\textbf{Lógica implementada:}
\begin{itemize}
    \item \textbf{Captura de dois números} do usuário
    \item \textbf{Seleção de operação} matemática
    \item \textbf{Validação de entrada} (divisão por zero)
    \item \textbf{Execução da operação} selecionada
    \item \textbf{Exibição do resultado} formatado
\end{itemize}

\textbf{Conceitos demonstrados:}
\begin{itemize}
    \item Estruturas de decisão (if/else)
    \item Switch case para múltiplas opções
    \item Tratamento de erros básico
    \item Lógica de cálculo matemático
\end{itemize}
\end{frame}

% Secao 3: Transicao: Console para WPF
\section{Transicao: Console para WPF}

\begin{frame}{O que muda ao ir para WPF?}
\textbf{Principais mudanças:}
\begin{itemize}
    \item \textbf{Saída de texto} → \textbf{Interface gráfica}
    \item \textbf{Entrada via teclado} → \textbf{Controles visuais}
    \item \textbf{Programa sequencial} → \textbf{Programa orientado a eventos}
    \item \textbf{Um arquivo .cs} → \textbf{Arquivos .cs + .xaml}
    \item \textbf{Execução linear} → \textbf{Execução baseada em interações}
\end{itemize}

\textbf{Novos conceitos:}
\begin{itemize}
    \item Programação orientada a eventos
    \item Separação entre interface e lógica
    \item Controles visuais e layout
    \item Gerenciamento de estado da interface
\end{itemize}
\end{frame}

\begin{frame}{Estrutura de um projeto WPF}
\textbf{Arquivos principais:}
\begin{itemize}
    \item \textbf{App.xaml} - Configuração global da aplicação
    \item \textbf{App.xaml.cs} - Código de inicialização e configuração
    \item \textbf{MainWindow.xaml} - Interface principal em linguagem declarativa
    \item \textbf{MainWindow.xaml.cs} - Lógica de negócio da janela principal
    \item \textbf{Properties/} - Configurações e recursos do projeto
\end{itemize}

\textbf{Organização:}
\begin{itemize}
    \item Separação clara entre interface (XAML) e lógica (C\#)
    \item Arquivos de configuração centralizados
    \item Estrutura hierárquica bem definida
\end{itemize}
\end{frame}

% Secao 4: Exemplo WPF: Calculadora com Interface Grafica
\section{Exemplo WPF: Calculadora com Interface Grafica}

\begin{frame}{Interface XAML da calculadora}
\textbf{Interface implementada:}
\begin{itemize}
    \item \textbf{Display numérico} para mostrar números e resultados
    \item \textbf{Botões de operação} para selecionar cálculos
    \item \textbf{Layout responsivo} com Grid e StackPanel
    \item \textbf{Design moderno} com espaçamento e margens
\end{itemize}

\textbf{Conceitos demonstrados:}
\begin{itemize}
    \item Separação entre interface (XAML) e lógica (C\#)
    \item Controles visuais do WPF
    \item Layout e posicionamento de elementos
    \item Estrutura de interface declarativa
\end{itemize}
\end{frame}

\begin{frame}{Logica C\# da calculadora}
\textbf{Funcionalidades implementadas:}
\begin{itemize}
    \item \textbf{Armazenamento de estado} (números e operações)
    \item \textbf{Manipulação de eventos} de clique nos botões
    \item \textbf{Lógica de cálculo} matemático
    \item \textbf{Atualização da interface} em tempo real
    \item \textbf{Gerenciamento de estado} da calculadora
\end{itemize}

\textbf{Conceitos demonstrados:}
\begin{itemize}
    \item Programação orientada a eventos
    \item Manipulação de controles WPF
    \item Lógica de aplicação desktop
    \item Gerenciamento de estado da interface
\end{itemize}
\end{frame}

% Secao 5: Exemplo WPF: Jogo de Combinacao
\section{Exemplo WPF: Jogo de Combinacao}

\begin{frame}{O que e o Jogo de Combinacao?}
\begin{itemize}
    \item \textbf{Objetivo:} Encontrar pares de cores iguais
    \item \textbf{Interface:} Grade de botoes coloridos
    \item \textbf{Mecanica:} Clicar em dois botoes para verificar
    \item \textbf{Vitoria:} Todos os pares encontrados
    \item \textbf{Tecnologias:} WPF + XAML + C\#
\end{itemize}
\end{frame}

\begin{frame}{Interface XAML do jogo}
\textbf{Estrutura da interface:}
\begin{itemize}
    \item \textbf{Título do jogo} centralizado e destacado
    \item \textbf{Contador de pares} encontrados
    \item \textbf{Grade de jogo} com botões organizados
    \item \textbf{Layout responsivo} que se adapta ao conteúdo
    \item \textbf{Design limpo} e focado na jogabilidade
\end{itemize}

\textbf{Conceitos demonstrados:}
\begin{itemize}
    \item Organização de elementos visuais
    \item Controles de layout avançados
    \item Estruturação hierárquica da interface
    \item Nomenclatura de elementos para acesso via código
\end{itemize}
\end{frame}

\begin{frame}{Logica C\# do jogo}
\textbf{Funcionalidades implementadas:}
\begin{itemize}
    \item \textbf{Criação dinâmica} do tabuleiro de jogo
    \item \textbf{Gerenciamento de estado} dos botões
    \item \textbf{Lógica de verificação} de pares
    \item \textbf{Controle de turnos} e seleções
    \item \textbf{Atualização visual} da interface
\end{itemize}

\textbf{Conceitos demonstrados:}
\begin{itemize}
    \item Criação dinâmica de controles
    \item Arrays bidimensionais
    \item Gerenciamento de eventos complexos
    \item Lógica de jogo interativo
\end{itemize}
\end{frame}

% Secao 6: Configuracao do Ambiente
\section{Configuracao do Ambiente}

\begin{frame}{O que voce precisa instalar?}
\textbf{Ferramentas essenciais:}
\begin{itemize}
    \item \textbf{Visual Studio 2022} (Community e gratuito)
    \item \textbf{.NET 8.0 SDK} (ou versão mais recente)
    \item \textbf{Workloads específicos:}
    \begin{itemize}
        \item .NET desktop development
        \item Desktop development with C\#
    \end{itemize}
\end{itemize}

\textbf{Extensões recomendadas:}
\begin{itemize}
    \item \textbf{C\# Dev Kit} - Desenvolvimento avançado
    \item \textbf{IntelliCode} - Inteligência artificial
    \item \textbf{Code Spell Checker} - Verificação ortográfica
\end{itemize}

\textbf{Vantagens:}
\begin{itemize}
    \item Ambiente completo e integrado
    \item Ferramentas gratuitas e poderosas
    \item Suporte oficial da Microsoft
\end{itemize}
\end{frame}

\begin{frame}{Como criar um projeto WPF}
\textbf{Passos para criação:}
\begin{enumerate}
    \item \textbf{File} → \textbf{New} → \textbf{Project}
    \item Selecionar \textbf{"WPF Application"}
    \item Definir \textbf{nome} e \textbf{localização}
    \item Escolher \textbf{.NET 8.0} (ou mais recente)
    \item Clicar em \textbf{Create}
    \item Projeto criado com estrutura básica
\end{enumerate}

\textbf{Configurações importantes:}
\begin{itemize}
    \item Framework .NET mais recente disponível
    \item Nome descritivo para o projeto
    \item Localização organizada no sistema
    \item Estrutura padrão do Visual Studio
\end{itemize}
\end{frame}

\begin{frame}{Estrutura do projeto criado}
\textbf{Arquivos e pastas:}
\begin{itemize}
    \item \textbf{App.xaml} - Configuração da aplicação
    \item \textbf{App.xaml.cs} - Código de inicialização
    \item \textbf{MainWindow.xaml} - Interface principal
    \item \textbf{MainWindow.xaml.cs} - Lógica da janela
    \item \textbf{Properties/} - Configurações do projeto
    \item \textbf{bin/} e \textbf{obj/} - Arquivos de compilação
\end{itemize}

\textbf{Propósito:}
\begin{itemize}
    \item Arquivos de interface declarativa (XAML)
    \item Código de lógica de negócio (C\#)
    \item Configurações e recursos do projeto
    \item Arquivos gerados automaticamente
\end{itemize}
\end{frame}

% Secao 7: Proximos Passos
\section{Proximos Passos}

\begin{frame}{O que vem depois desta aula?}
\textbf{Próximos passos:}
\begin{itemize}
    \item \textbf{Prática:} Implementar os exemplos completos
    \item \textbf{Conceitos:} Classes, objetos, herança
    \item \textbf{Controles:} Mais elementos da interface
    \item \textbf{Eventos:} Outros tipos de interação
    \item \textbf{Projeto:} Aplicação mais complexa
    \item \textbf{Deploy:} Distribuir a aplicação
\end{itemize}

\textbf{Objetivos:}
\begin{itemize}
    \item Consolidar conceitos fundamentais
    \item Aplicar conhecimentos em projetos práticos
    \item Evoluir para aplicações mais complexas
    \item Preparar para desenvolvimento profissional
\end{itemize}
\end{frame}

\begin{frame}{Recursos para estudo}
\textbf{Materiais recomendados:}
\begin{itemize}
    \item \textbf{Livro:} "Use a Cabeça C\#" (Quarta Edição)
    \item \textbf{Documentação:} Microsoft Learn
    \item \textbf{Vídeos:} Canais especializados em C\#
    \item \textbf{Comunidade:} Stack Overflow, Reddit
    \item \textbf{Prática:} Exercícios e projetos pessoais
    \item \textbf{Projetos open source:} GitHub
\end{itemize}

\textbf{Estratégias de aprendizado:}
\begin{itemize}
    \item Teoria + prática simultânea
    \item Projetos pessoais para aplicar conceitos
    \item Participação ativa na comunidade
    \item Estudo contínuo e atualizado
\end{itemize}
\end{frame}

% Secao 8: Resumo
\section{Resumo}

\begin{frame}{Resumo da Aula}
\textbf{Pontos principais:}
\begin{itemize}
    \item \textbf{Console → WPF:} Progressão natural de aprendizado
    \item \textbf{Conceitos fundamentais:} Estrutura, sintaxe e lógica
    \item \textbf{C\#:} Linguagem poderosa e versátil
    \item \textbf{Prática:} Aprender fazendo é fundamental
    \item \textbf{Próximo:} Implementar os exemplos completos!
\end{itemize}

\textbf{Conceitos aprendidos:}
\begin{itemize}
    \item Estrutura básica de programas C\#
    \item Transição de Console para WPF
    \item Programação orientada a eventos
    \item Separação entre interface e lógica
\end{itemize}
\end{frame}

\begin{frame}{Obrigado!}
\begin{center}
\Large \textbf{Dúvidas?}
\vspace{1cm}
\normalsize
\textbf{Próxima aula:} Implementação prática dos exemplos
\end{center}

\textbf{Contato:}
\begin{itemize}
    \item \textbf{Professor:} Bento Rafael Siqueira
    \item \textbf{Disciplina:} GCT052 - Programação Orientada a Objetos
    \item \textbf{Universidade:} UFLA
\end{itemize}
\end{frame}

\end{document}

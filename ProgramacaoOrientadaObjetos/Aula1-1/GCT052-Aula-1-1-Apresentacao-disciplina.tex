\documentclass[aspectratio=169]{beamer}

% Configurações do tema
\usetheme{Berkeley}
\usecolortheme{spruce}
\setbeamertemplate{navigation symbols}{}
\setbeamertemplate{footline}[frame number]

% Configuração do logo da UFLA
\setbeamertemplate{headline}{%
  \begin{beamercolorbox}[wd=\paperwidth,ht=1.2cm,dp=0.2cm]{section in head/foot}%
    \begin{center}
      \vspace{0.1cm}
      \includegraphics[height=1.0cm]{../../Image/ufla-logo.PNG}
    \end{center}
  \end{beamercolorbox}%
}

% Configuração do título centralizado
\setbeamertemplate{frametitle}{%
  \begin{beamercolorbox}[wd=\paperwidth,ht=1.2cm,dp=0.3cm]{frametitle}%
    \begin{center}
      \insertframetitle
    \end{center}
  \end{beamercolorbox}%
}

% Pacotes necessários
\usepackage[utf8]{inputenc}
\usepackage[brazilian]{babel}
\usepackage{amsmath}
\usepackage{amsfonts}
\usepackage{amssymb}
\usepackage{graphicx}
\usepackage{hyperref}
\usepackage{multicol}

% Configurações de cores
\definecolor{azul}{RGB}{0,51,102}
\definecolor{azulclaro}{RGB}{51,102,153}
\setbeamercolor{title}{fg=white,bg=azul}
\setbeamercolor{frametitle}{fg=white,bg=azulclaro}
\setbeamercolor{section title}{fg=white,bg=azul}

% Informações do documento
\title{Programação Orientada a Objetos}
\subtitle{GCT052 - Aula 1.1 - Apresentação da Disciplina}
\author{Bento Rafael Siqueira}
\institute{Universidade Federal de Lavras (UFLA)}
\date{\today}

\begin{document}

% Slide de título
\begin{frame}
\titlepage
\end{frame}

% Slide de sumário
\begin{frame}{Sumário}
\tableofcontents
\end{frame}

% Seção 1: Informações Gerais
\section{Informações Gerais}

\begin{frame}{Informações Gerais}
\begin{itemize}
    \item \textbf{Disciplina:} Programação Orientada a Objetos
    \item \textbf{Código:} GCT052
    \item \textbf{Carga Horária:} 68 horas (34 Teórica, 34 Prática) 17 semanas
    \item \textbf{Aulas Semanais:} 4 aulas
    \item \textbf{Pré-requisitos:} Programação Básica
\end{itemize}
\end{frame}

% Seção 2: Objetivos
\section{Objetivos}

\begin{frame}{Objetivos Gerais}
\begin{itemize}
    \item Compreender os fundamentos da programação orientada a objetos
    \item Aplicar conceitos de modelagem com classes, objetos, métodos e herança
    \item Desenvolver sistemas utilizando boas práticas de programação
    \item Estimular o raciocínio lógico e trabalho em equipe
\end{itemize}
\end{frame}

\begin{frame}{Objetivos Específicos}
\begin{itemize}
    \item Dominar os conceitos de encapsulamento, herança e polimorfismo
    \item Implementar soluções usando classes abstratas e interfaces
    \item Aplicar tratamento de exceções em programas orientados a objetos
    \item Desenvolver projetos seguindo padrões de projeto e boas práticas
    \item Dominar o uso de ferramentas de desenvolvimento de software
\end{itemize}
\end{frame}

% Seção 3: Ementa
\section{Ementa}

\begin{frame}{Ementa}
A disciplina aborda os conceitos fundamentais de programação orientada a objetos, incluindo:
\begin{itemize}
    \item Visão geral sobre Paradigmas de Programação. 
    \item Conceitos Básicos de Programação Orientada a Objetos. 
    \item Encapsulamento, classe, objeto, atributo, método, construtor, atributos e métodos estáticos, composição e agregação, herança simples, sobrecarga e sobrescrita. 
    \item Conceitos avançados de Orientação a Objetos. Herança múltipla, polimorfismo, classes abstratas e interfaces, tratamento de exceção e diagrama de classes UML. 
    \item Projetos de implementação utilizando linguagem de programação Orientada a Objetos abordando conceitos apresentados na disciplina.
\end{itemize}
\end{frame}

% Seção 4: Conteúdo Programático
\section{Conteúdo Programático}

\begin{frame}{Conteúdo Programático}
\begin{columns}
\begin{column}{0.5\textwidth}
\begin{enumerate}
    \item Introdução com hands on
    \item Declarações e classes
    \item Unity - Lab 1
    \item Objetos
    \item Tipos e Referências
    \item Unity - Lab 2
    \item Encapsulamento
    \item Herança
    \item Unity - Lab 3
    \item Interfaces
\end{enumerate}
\end{column}
\begin{column}{0.5\textwidth}
\begin{enumerate}
    \setcounter{enumi}{10}
    \item Enums e Coleções
    \item Unity - Lab 4
    \item LINQ / Lambdas
    \item Arquivos
    \item Unity - Lab 5
    \item Ciclo de vida do objeto
    \item Tratamento de Exceções
    \item Unity - Lab 6
    \item Desenvolvimento Web
    \item Projeto - ONG de Animais
\end{enumerate}
\end{column}
\end{columns}
\end{frame}

% Seção 5: Tecnologias e Ferramentas
\section{Tecnologias e Ferramentas}

\begin{frame}{Tecnologias e Ferramentas}
\begin{columns}
\begin{column}{0.5\textwidth}
\textbf{Ambiente de Desenvolvimento:}
\begin{itemize}
    \item Visual Studio Community 2022
    \item Visual Studio Code
    \item .NET 8.0 SDK
    \item .NET Core 6.0/7.0
\end{itemize}

\textbf{Desktop:}
\begin{itemize}
    \item Windows Presentation Foundation (WPF)
    \item Windows Forms
    \item Console
\end{itemize}
\end{column}
\begin{column}{0.5\textwidth}
\textbf{Web \& Cloud:}
\begin{itemize}
    \item ASP.NET Core 8.0
    \item Blazor (Server \& WebAssembly) 8.0
    \item Azure Functions
\end{itemize}

\textbf{Jogos \& 3D:}
\begin{itemize}
    \item Unity 2022.3 LTS
    \item Unity 2023.2 LTS
    \item MonoGame
\end{itemize}
\end{column}
\end{columns}
\end{frame}

\begin{frame}{Ferramentas de Modelagem UML}
\begin{columns}
\begin{column}{0.5\textwidth}
\textbf{Ferramentas Profissionais:}
\begin{itemize}
    \item Visual Studio UML Designer
    \item Lucidchart
    \item StarUML
    \item Enterprise Architect
    \item Visual Paradigm
\end{itemize}
\end{column}
\begin{column}{0.5\textwidth}
\textbf{Ferramentas Gratuitas:}
\begin{itemize}
    \item Draw.io (diagrams.net)
    \item PlantUML
    \item Mermaid
    \item ArgoUML
    \item IBM Rational (comunidade)
\end{itemize}
\end{column}
\end{columns}
\end{frame}

% Seção 6: Bibliografia
\section{Bibliografia}

\begin{frame}{Bibliografia Básica}
\begin{itemize}
    \item \textbf{HEAD FIRST.} Use a cabeça! C\#. Alta Books.
    \item Perkovic, L. Introdução à computação usando Python : um foco no desenvolvimento de aplicações - 1. ed. Rio de Janeiro: LTC, 2016. ISBN 9788521630814.
    \item KÖLLING, M. Programação orientada a objetos com Java: uma introdução prática usando o Bluej. 1. ed. São Paulo, SP: Pearson, 2004. ISBN 9788576050124.
    \item GAMMA, Erich et al. Padrões de projeto: soluções reutilizáveis de software orientado a objetos. Porto Alegre, RS: Bookman, 2011. xii, 364 p. ISBN 9788577800469.
\end{itemize}
\end{frame}

% Seção 7: Critérios de Avaliação
\section{Critérios de Avaliação}

\begin{frame}{Critérios de Avaliação}
\begin{itemize}
    \item \textbf{Prova 1 (P1):} 25\%
    \item \textbf{Prova 2 (P2):} 25\%
    \item \textbf{Aula de Atividade (AA):} 20\%
    \item \textbf{Projeto em Grupo (PG):} 30\% (3 a 4 pessoas)
\end{itemize}

\vspace{0.5cm}
\textbf{Nota Final = P1 + P2 + AA + PG}
\end{frame}

% Seção 8: Cronograma Geral
\section{Cronograma Geral}

\begin{frame}{Cronograma Geral}
\begin{itemize}
    \item \textbf{Semanas 1-2:} Introdução com hands on e Declarações e classes
    \item \textbf{Semanas 3-4:} Unity - Lab 1 e Objetos
    \item \textbf{Semanas 5-6:} Tipos e Referências e Unity - Lab 2
    \item \textbf{Semanas 7-8:} Encapsulamento e Herança
    \item \textbf{Semanas 9-10:} Unity - Lab 3 e Interfaces
    \item \textbf{Semanas 11-12:} Enums e Coleções e Unity - Lab 4
    \item \textbf{Semanas 13-14:} LINQ/Lambdas e Arquivos
    \item \textbf{Semanas 15-16:} Unity - Lab 5, Ciclo de vida do objeto e Tratamento de Exceções
    \item \textbf{Semana 17:} Unity - Lab 6, Desenvolvimento Web e Projeto - ONG de Animais
\end{itemize}
\end{frame}

% Seção 9: Informações Adicionais
\section{Informações Adicionais}

\begin{frame}{Horário de Atendimento}
\begin{itemize}
    \item \textbf{Dia:} Segunda-feira
    \item \textbf{Horário:} 16:00 às 17:00
    \item \textbf{Email:} bento.siqueira@ufla.br
    \item \textbf{Local:} Sala PAV1-109
\end{itemize}
\end{frame}

\begin{frame}{Metodologia}
\begin{itemize}
    \item Aulas expositivas com exemplos práticos
    \item Exercícios em sala de aula
    \item Desenvolvimento de projetos práticos
    \item Trabalho em equipe
    \item Uso de ferramentas de desenvolvimento
\end{itemize}
\end{frame}

% Slide final
\begin{frame}{Dúvidas?}
\begin{center}
\Large
\textbf{Obrigado pela atenção!}

\vspace{1cm}
\textbf{Programação Orientada a Objetos}

\vspace{0.5cm}
GCT052 - 2025
\end{center}
\end{frame}

\end{document}

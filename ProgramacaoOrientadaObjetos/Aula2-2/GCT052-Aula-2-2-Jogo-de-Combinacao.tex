\documentclass[aspectratio=169]{beamer}

% Configuracoes do tema
\usetheme{Berkeley}
\usecolortheme{spruce}
\setbeamertemplate{navigation symbols}{}
\setbeamertemplate{footline}[frame number]

% Configuracoes para evitar overflow
\setbeamersize{text margin left=0.5cm}
\setbeamersize{text margin right=0.5cm}
\setbeamertemplate{itemize items}[circle]
\setbeamertemplate{enumerate items}[default]

% Configuracoes de fonte para evitar overflow
\setbeamerfont{itemize item}{size=\footnotesize}
\setbeamerfont{itemize subitem}{size=\tiny}
\setbeamerfont{enumerate item}{size=\footnotesize}

% Configuracoes especificas para o sumario
\setbeamertemplate{section in toc}[sections numbered]
\setbeamertemplate{subsection in toc}[subsections numbered]

% Configuracao do logo da UFLA (compacto)
\setbeamertemplate{headline}{%
  \begin{beamercolorbox}[wd=\paperwidth,ht=0.8cm,dp=0.1cm]{section in head/foot}%
    \begin{center}
      \vspace{0.05cm}
      \includegraphics[height=0.7cm]{../../Image/ufla-logo.PNG}
    \end{center}
  \end{beamercolorbox}%
}

% Configuracao do titulo centralizado e compacto
\setbeamertemplate{frametitle}{%
  \begin{beamercolorbox}[wd=\paperwidth,ht=0.8cm,dp=0.2cm]{frametitle}%
    \begin{center}
      \insertframetitle
    \end{center}
  \end{beamercolorbox}%
}

% Pacotes necessarios para caracteres especiais - UTF-8
\usepackage[utf8]{inputenc}
\usepackage[T1]{fontenc}
% \usepackage{newunicodechar}


\usepackage[brazilian]{babel}
\usepackage{amsmath}
\usepackage{amsfonts}
\usepackage{amssymb}
\usepackage{graphicx}
\usepackage{hyperref}
\usepackage{multicol}
\usepackage{tikz}

% Configuracoes de cores
\definecolor{azul}{RGB}{0,51,102}
\definecolor{azulclaro}{RGB}{51,102,153}
\definecolor{verde}{RGB}{34,139,34}
\definecolor{laranja}{RGB}{255,140,0}
\setbeamercolor{title}{fg=white,bg=azul}
\setbeamercolor{frametitle}{fg=white,bg=azulclaro}
\setbeamercolor{section title}{fg=white,bg=azul}

% Informacoes do documento
\title{Programacao Orientada a Objetos}
\subtitle{GCT052 - Aula 2.2 - Jogo de Combinação (Animal Matching Game)}
\author{Bento Rafael Siqueira}
\institute{Universidade Federal de Lavras (UFLA)}
\date{\today}

\begin{document}

% Slide de titulo
\begin{frame}
\titlepage
\end{frame}

% Sumario da apresentacao
\begin{frame}{Sumario}
\textbf{Conteudo da aula:}
\begin{itemize}
    \item Introducao ao Jogo de Combinacao
    \item Estrutura do Projeto
    \item Passo a Passo da Implementacao
    \item Detalhes da Grade 4x4
    \item Conceitos Aplicados
    \item Referencias aos Codigos
    \item Recursos e Referencias
    \item Resumo
\end{itemize}
\end{frame}

% Secao 1: Introducao ao Jogo de Combinacao
\section{Introducao ao Jogo de Combinacao}

\begin{frame}{O que vamos aprender hoje?}
\textbf{Objetivos da aula:}
\begin{itemize}
    \item \textbf{Implementar} o Jogo de Combinação completo
    \item \textbf{Aplicar} conceitos de WPF e XAML
    \item \textbf{Desenvolver} lógica de jogo interativo
    \item \textbf{Praticar} programação orientada a eventos
\end{itemize}

\textbf{Base:} Head First C\# 4th Edition - Animal Matching Game
\end{frame}

\begin{frame}{O que e o Jogo de Combinacao?}
\textbf{Descrição do jogo:}
\begin{itemize}
    \item \textbf{Objetivo:} Encontrar pares de animais iguais
    \item \textbf{Interface:} Grade 4x4 simples
    \item \textbf{Mecânica:} Clicar em duas cartas para verificar
    \item \textbf{Vitória:} Todos os 8 pares encontrados
\end{itemize}

\textbf{Conceitos básicos:} TextBlock, Button, Eventos
\end{frame}

% Secao 2: Estrutura do Projeto
\section{Estrutura do Projeto}

\begin{frame}{Estrutura do projeto WPF}
\textbf{Arquivos principais:}
\begin{itemize}
    \item \textbf{MainWindow.xaml} - Interface simples do jogo
    \item \textbf{MainWindow.xaml.cs} - Lógica básica (100 linhas)
    \item \textbf{App.xaml} - Configuração mínima
    \item \textbf{App.xaml.cs} - Inicialização padrão
\end{itemize}

\textbf{Organização:} Interface (XAML) + Lógica (C\#)
\end{frame}

\begin{frame}{Interface XAML - Estrutura basica}
\textbf{Layout principal:}
\begin{itemize}
    \item \textbf{Grid} como container principal (3 linhas)
    \item \textbf{TextBlock} para título do jogo
    \item \textbf{TextBlock} para status (StatusText)
    \item \textbf{Grid} para grade 4x4 das cartas (GameGrid)
    \item \textbf{Button} para novo jogo (RestartButton)
\end{itemize}

\textbf{Design:} Interface simples e clara
\end{frame}

% Secao 3: Passo a Passo da Implementacao
\section{Passo a Passo da Implementacao}

\begin{frame}{Passo 1: Criar o projeto WPF}
\textbf{Configuração inicial:}
\begin{enumerate}
    \item \textbf{File} → \textbf{New} → \textbf{Project}
    \item Selecionar \textbf{"WPF Application"}
    \item Nome: \textbf{"AnimalMatchingGame"}
    \item Framework: \textbf{.NET 9.0}
    \item Clicar em \textbf{Create}
\end{enumerate}

\textbf{Verificação:}
\begin{itemize}
    \item Projeto criado com estrutura básica
    \item MainWindow.xaml e MainWindow.xaml.cs presentes
    \item Ambiente pronto para desenvolvimento
\end{itemize}
\end{frame}

\begin{frame}{Passo 2: Configurar a interface XAML}
\textbf{Elementos da interface:}
\begin{itemize}
    \item \textbf{Título:} "Animal Matching Game"
    \item \textbf{Contador:} "Pares encontrados: 0/8"
    \item \textbf{Grade:} 4x4 para os emojis de animais
    \item \textbf{Botão:} "Novo Jogo" para reiniciar
\end{itemize}

\textbf{Layout implementado:}
\begin{itemize}
    \item Grid com 3 linhas principais
    \item UniformGrid para organização dos emojis
    \item Controles posicionados adequadamente
    \item Estilo visual consistente
\end{itemize}
\end{frame}

\begin{frame}{Passo 3: Definir os dados do jogo}
\textbf{Lista de animais:}
\begin{itemize}
    \item \textbf{8 pares} de emojis de animais
    \item \textbf{16 posições} na grade 4x4
    \item \textbf{Emojis:} Cachorro, Gato, Rato, Hamster, Coelho, Raposa, Urso, Panda
    \item \textbf{Distribuição:} Cada animal aparece 2 vezes
\end{itemize}

\textbf{Estrutura:} Arrays, Listas e Dicionários
\end{frame}

\begin{frame}{Passo 4: Definir controles estáticos no XAML}
\textbf{Grade estática 4x4:}
\begin{itemize}
    \item \textbf{Grid.RowDefinitions} e \textbf{Grid.ColumnDefinitions} (4x4)
    \item \textbf{16 TextBlocks} predefinidos (TextBlock00 a TextBlock33)
    \item \textbf{Evento MouseLeftButtonDown} associado a cada TextBlock
    \item \textbf{Posicionamento} fixo no Grid
\end{itemize}

\textbf{Sistema de numeração:} [Linha][Coluna]

\textbf{Grade 4x4 simplificada:}
\begin{center}
\begin{tabular}{|c|c|c|c|}
\hline
00 & 01 & 02 & 03 \\
\hline
10 & 11 & 12 & 13 \\
\hline
20 & 21 & 22 & 23 \\
\hline
30 & 31 & 32 & 33 \\
\hline
\end{tabular}
\end{center}

\textbf{Convenção:} Primeiro dígito = Linha, Segundo dígito = Coluna
\end{frame}

\begin{frame}{Detalhes da Grade 4x4 - Sistema de Coordenadas}
\tiny
\textbf{Explicação detalhada da numeração:}

\begin{center}
\begin{tabular}{|c|c|c|c|c|}
\hline
\textbf{Posição} & \textbf{TextBlock} & \textbf{Linha} & \textbf{Coluna} & \textbf{Coordenadas} \\
\hline
00 & TextBlock00 & 0 & 0 & (0,0) \\
01 & TextBlock01 & 0 & 1 & (0,1) \\
02 & TextBlock02 & 0 & 2 & (0,2) \\
03 & TextBlock03 & 0 & 3 & (0,3) \\
10 & TextBlock10 & 1 & 0 & (1,0) \\
11 & TextBlock11 & 1 & 1 & (1,1) \\
12 & TextBlock12 & 1 & 2 & (1,2) \\
13 & TextBlock13 & 1 & 3 & (1,3) \\
20 & TextBlock20 & 2 & 0 & (2,0) \\
21 & TextBlock21 & 2 & 1 & (2,1) \\
22 & TextBlock22 & 2 & 2 & (2,2) \\
23 & TextBlock23 & 2 & 3 & (2,3) \\
30 & TextBlock30 & 3 & 0 & (3,0) \\
31 & TextBlock31 & 3 & 1 & (3,1) \\
32 & TextBlock32 & 3 & 2 & (3,2) \\
33 & TextBlock33 & 3 & 3 & (3,3) \\
\hline
\end{tabular}
\end{center}

\small
\textbf{Características dos TextBlocks:}
\begin{itemize}
    \item \textbf{FontSize:} 24
    \item \textbf{Background:} LightBlue
    \item \textbf{Margin:} 2
    \item \textbf{Evento:} MouseLeftButtonDown
\end{itemize}
\end{frame}

\begin{frame}{Passo 5: Implementar a lógica do jogo}
\textbf{Funcionalidades principais:}
\begin{itemize}
    \item \textbf{Seleção de cartas} (máximo 2 por vez)
    \item \textbf{Verificação de pares} (animais iguais)
    \item \textbf{Controle de estado} (primeira/segunda carta)
    \item \textbf{Atualização do status} de pares
\end{itemize}

\textbf{Lógica:} TextBlock\_Click, verificação simples
\end{frame}

\begin{frame}{Passo 6: Gerenciar eventos de clique}
\textbf{Estados das cartas:}
\begin{itemize}
    \item \textbf{Oculto:} LightBlue, Text="?" (estado inicial)
    \item \textbf{Revelado:} LightYellow, Text=animal (selecionado)
    \item \textbf{Encontrado:} LightGreen, Text=animal (par completo)
\end{itemize}

\textbf{Controle:} Verificação simples se Text != "?"
\end{frame}

\begin{frame}{Passo 7: Implementar verificação de pares}
\textbf{Algoritmo de verificação:}
\begin{itemize}
    \item \textbf{Primeiro clique:} Revela a carta (primeiraCarta)
    \item \textbf{Segundo clique:} Revela e compara (segundaCarta)
    \item \textbf{Se iguais:} Marca como encontrado (verde)
    \item \textbf{Se diferentes:} Oculta após 1 segundo (timer)
\end{itemize}

\textbf{Timer:} DispatcherTimer simples com delay de 1 segundo
\end{frame}

\begin{frame}{Passo 8: Adicionar funcionalidades extras}
\textbf{Recursos adicionais:}
\begin{itemize}
    \item \textbf{Botão "Novo Jogo"} para reiniciar (RestartButton\_Click)
    \item \textbf{Status de pares} encontrados (StatusText)
    \item \textbf{Mensagem de vitória} quando completar (MessageBox)
    \item \textbf{Embaralhamento} das cartas (Random + LINQ)
\end{itemize}

\textbf{Melhorias:} Interface simples e funcional
\end{frame}

% Secao 4: Conceitos Aplicados
\section{Conceitos Aplicados}

\begin{frame}{Conceitos de C\# aplicados}
\textbf{Elementos da linguagem:}
\begin{itemize}
    \item \textbf{Arrays e Listas} para armazenar dados
    \item \textbf{Loops} para criar controles dinamicamente
    \item \textbf{Condicionais} para verificar pares
    \item \textbf{Métodos} para organizar o código
\end{itemize}

\textbf{Conceitos avançados:}
\begin{itemize}
    \item \textbf{Eventos} para interação do usuário
    \item \textbf{Delegates} para callbacks
    \item \textbf{Timers} para controle de tempo
    \item \textbf{Threading} para operações assíncronas
\end{itemize}
\end{frame}

\begin{frame}{Conceitos de WPF aplicados}
\textbf{Controles utilizados:}
\begin{itemize}
    \item \textbf{Grid} para layout principal
    \item \textbf{TextBlock} para grade de cartas
    \item \textbf{TextBlock} para texto e contadores
    \item \textbf{Button} para interações
\end{itemize}

\textbf{Recursos do WPF:}
\begin{itemize}
    \item \textbf{Event Handling} para interações
    \item \textbf{Layout} responsivo com Grid
    \item \textbf{Properties} para configuração visual
    \item \textbf{Eventos} MouseLeftButtonDown
\end{itemize}
\end{frame}

\begin{frame}{Conceitos de POO aplicados}
\textbf{Princípios utilizados:}
\begin{itemize}
    \item \textbf{Encapsulamento} em classes e métodos
    \item \textbf{Separação de responsabilidades} (UI vs lógica)
    \item \textbf{Reutilização} de código
    \item \textbf{Manutenibilidade} do código
\end{itemize}

\textbf{Padrões:} Event-driven programming, State management
\end{frame}

% Secao 4.5: Referencias aos Codigos Implementados
\section{Referencias aos Codigos Implementados}

\begin{frame}{Estrutura de Dados - Arrays e Listas}
\textbf{Definição dos dados do jogo:}

\textbf{Elementos principais:}
\begin{itemize}
    \item \textbf{Array:} animais[] com 8 animais únicos
    \item \textbf{Lista:} cartas com 16 elementos (8 pares)
    \item \textbf{Variáveis de estado:} primeiraCarta, segundaCarta
    \item \textbf{Array 2D:} grade[4,4] para mapear TextBlocks
    \item \textbf{Timer:} timer para controle de tempo
\end{itemize}

\textbf{Código implementado:}
\begin{itemize}
    \item \textbf{Arquivo:} MainWindow.xaml.cs (linhas 15-25)
    \item \textbf{Array:} private string[] animais
    \item \textbf{Lista:} private List<string> cartas
    \item \textbf{Variáveis:} private TextBlock primeiraCarta, segundaCarta
    \item \textbf{Array:} private TextBlock[,] grade
    \item \textbf{Timer:} private DispatcherTimer timer
\end{itemize}
\end{frame}

\begin{frame}{Método SetupGame - Configuração Inicial}
\textbf{Configuração de TextBlocks estáticos e embaralhamento:}

\textbf{Funcionalidades implementadas:}
\begin{itemize}
    \item \textbf{Inicialização:} InitializeGameTextBlocks() mapeia TextBlocks XAML
    \item \textbf{Array 2D:} TextBlock[,] grade para acesso
    \item \textbf{Lista de cartas:} cartas com duplicatas
    \item \textbf{Embaralhamento:} Random + LINQ OrderBy
    \item \textbf{Distribuição:} Animais nos TextBlocks estáticos
    \item \textbf{Configuração:} Text="?", Tag=animal, Background=LightBlue
\end{itemize}

\textbf{Código implementado:}
\begin{itemize}
    \item \textbf{Arquivo:} MainWindow.xaml.cs (linhas 27-60)
    \item \textbf{Método:} private void SetupGame()
    \item \textbf{Inicialização:} InitializeGameTextBlocks()
    \item \textbf{Embaralhamento:} cartas.OrderBy(x => random.Next())
    \item \textbf{Distribuição:} grade[row, col].Tag = animal
\end{itemize}
\end{frame}

\begin{frame}{Método InitializeGameTextBlocks - Mapeamento Estático}
\textbf{Mapeamento de TextBlocks XAML para array:}

\textbf{Características do mapeamento:}
\begin{itemize}
    \item \textbf{Array 2D:} TextBlock[,] grade[4,4]
    \item \textbf{Mapeamento:} TextBlock00 → grade[0,0]
    \item \textbf{Acesso:} grade[row, col] para cada posição
    \item \textbf{Configuração:} Todos os 16 TextBlocks mapeados
    \item \textbf{Referência:} Acesso direto aos controles XAML
    \item \textbf{Organização:} Estrutura lógica por linha/coluna
\end{itemize}

\textbf{Código implementado:}
\begin{itemize}
    \item \textbf{Arquivo:} MainWindow.xaml.cs (linhas 27-45)
    \item \textbf{Método:} private void InitializeGameTextBlocks()
    \item \textbf{Array:} grade = new TextBlock[4, 4]
    \item \textbf{Mapeamento:} grade[0,0] = TextBlock00
    \item \textbf{Organização:} Todos os 16 TextBlocks mapeados
\end{itemize}
\end{frame}

\begin{frame}{Evento TextBlock\_MouseLeftButtonDown - Lógica Principal}
\textbf{Evento principal do jogo:}

\textbf{Lógica implementada:}
\begin{itemize}
    \item \textbf{Verificação de estado:} Text != "?" para verificar se já foi revelado
    \item \textbf{Revelação:} Tag.ToString() para mostrar animal
    \item \textbf{Controle de seleção:} primeiraCarta, segundaCarta
    \item \textbf{Verificação de pares:} Comparação de Tag
    \item \textbf{Chamada de métodos:} Verificação simples de igualdade
\end{itemize}

\textbf{Código implementado:}
\begin{itemize}
    \item \textbf{Arquivo:} MainWindow.xaml.cs (linhas 62-95)
    \item \textbf{Método:} private void TextBlock\_MouseLeftButtonDown(object sender, MouseButtonEventArgs e)
    \item \textbf{Verificação:} if (clickedTextBlock.Text != "?") return;
    \item \textbf{Revelação:} clickedTextBlock.Text = animal
    \item \textbf{Comparação:} if (primeiraCarta.Tag.ToString() == segundaCarta.Tag.ToString())
    \item \textbf{Ações:} Marcação de pares e atualização de status
\end{itemize}
\end{frame}

\begin{frame}{Processamento de Pares - Lógica Simplificada}
\textbf{Processamento quando um par é encontrado:}

\textbf{Ações realizadas:}
\begin{itemize}
    \item \textbf{Mudança de cor:} LightGreen para pares encontrados
    \item \textbf{Incremento:} paresEncontrados++ para contador
    \item \textbf{Limpeza:} primeiraCarta = null, segundaCarta = null
    \item \textbf{Verificação de vitória:} paresEncontrados == 8
    \item \textbf{Chamada:} MessageBox.Show() se jogo terminou
\end{itemize}

\textbf{Código implementado:}
\begin{itemize}
    \item \textbf{Arquivo:} MainWindow.xaml.cs (linhas 85-95)
    \item \textbf{Lógica:} Dentro do evento TextBlock\_MouseLeftButtonDown
    \item \textbf{Cores:} primeiraCarta.Background = LightGreen
    \item \textbf{Contador:} paresEncontrados++;
    \item \textbf{Limpeza:} primeiraCarta = null; segundaCarta = null;
    \item \textbf{Vitória:} if (paresEncontrados == 8) MessageBox.Show("Parabéns!");
\end{itemize}
\end{frame}

\begin{frame}{Timer - Controle de Tempo Simplificado}
\textbf{Timer para esconder cartas não combinadas:}

\textbf{Funcionalidades do timer:}
\begin{itemize}
    \item \textbf{Esconder cartas:} Text = "?", Background = LightBlue
    \item \textbf{Limpeza de seleção:} Reset das variáveis primeiraCarta/segundaCarta
    \item \textbf{Controle de estado:} Reset das variáveis de controle
    \item \textbf{Parar timer:} timer.Stop() para finalizar
    \item \textbf{Delay:} 1 segundo para visualização
\end{itemize}

\textbf{Código implementado:}
\begin{itemize}
    \item \textbf{Arquivo:} MainWindow.xaml.cs (linhas 97-110)
    \item \textbf{Método:} private void Timer\_Tick(object sender, EventArgs e)
    \item \textbf{Esconder:} primeiraCarta.Text = "?"; primeiraCarta.Background = LightBlue
    \item \textbf{Limpeza:} primeiraCarta = null; segundaCarta = null;
    \item \textbf{Parar:} timer.Stop();
\end{itemize}
\end{frame}

\begin{frame}{Interface XAML - MainWindow.xaml}
\textbf{Estrutura da interface gráfica:}

\textbf{Elementos da interface:}
\begin{itemize}
    \item \textbf{Window:} Título, dimensões, centralização
    \item \textbf{Grid principal:} 3 linhas (Auto, *, Auto)
    \item \textbf{Título:} TextBlock com nome do jogo
    \item \textbf{Área do jogo:} Grid 4x4 com TextBlocks
    \item \textbf{Status:} TextBlock para mostrar pares encontrados
    \item \textbf{Botão:} Button para novo jogo
\end{itemize}

\textbf{Código implementado:}
\begin{itemize}
    \item \textbf{Arquivo:} MainWindow.xaml (linhas 1-50)
    \item \textbf{Window:} Title="Jogo de Combinação"
    \item \textbf{Grid:} Grid.RowDefinitions com 3 linhas
    \item \textbf{Título:} TextBlock Text="Jogo de Combinação"
    \item \textbf{Área do jogo:} Grid 4x4 com 16 TextBlocks
    \item \textbf{Controles:} x:Name="StatusText", x:Name="RestartButton"
\end{itemize}
\end{frame}



% Secao 7: Recursos e Referencias
\section{Recursos e Referencias}

\begin{frame}{Recursos para estudo}
\textbf{Materiais recomendados:}
\begin{itemize}
    \item \textbf{Livro:} "Head First C\#" 4th Edition
    \item \textbf{Repositório:} github.com/head-first-csharp/fourth-edition
    \item \textbf{Documentação:} Microsoft Learn - WPF
    \item \textbf{Vídeos:} Canais especializados em C\#
\end{itemize}

\textbf{Comunidade e suporte:}
\begin{itemize}
    \item \textbf{Stack Overflow} para dúvidas técnicas
    \item \textbf{GitHub} para exemplos de código
    \item \textbf{Reddit} para discussões
    \item \textbf{Discord} para networking
\end{itemize}
\end{frame}

\begin{frame}{Próximos passos}
\textbf{O que vem depois:}
\begin{itemize}
    \item \textbf{Prática:} Implementar o jogo completo
    \item \textbf{Conceitos:} Herança e polimorfismo
    \item \textbf{Unity:} Lab 1 - Introdução ao Unity
    \item \textbf{Projeto:} Aplicação mais complexa
\end{itemize}

\textbf{Preparação:}
\begin{itemize}
    \item Revisar conceitos de WPF e XAML
    \item Praticar com eventos e controles
    \item Explorar recursos do Visual Studio
    \item Estudar padrões de design
\end{itemize}
\end{frame}

% Secao 8: Resumo
\section{Resumo}

\begin{frame}{Resumo da Aula}
\textbf{Pontos principais:}
\begin{itemize}
    \item \textbf{Implementação completa} do Animal Matching Game
    \item \textbf{Conceitos práticos} de WPF e C\#
    \item \textbf{Lógica de jogo} interativo
    \item \textbf{Programação orientada a eventos}
\end{itemize}

\textbf{Conceitos aprendidos:}
\begin{itemize}
    \item Criação dinâmica de controles
    \item Gerenciamento de estado de aplicação
    \item Manipulação de eventos complexos
    \item Separação entre interface e lógica
\end{itemize}
\end{frame}

\begin{frame}{Obrigado!}
\begin{center}
\Large \textbf{Dúvidas?}
\vspace{1cm}
\normalsize
\textbf{Próxima aula:} Unity Lab 1 - Introdução ao Unity
\end{center}

\textbf{Contato:}
\begin{itemize}
    \item \textbf{Professor:} Bento Rafael Siqueira
    \item \textbf{Disciplina:} GCT052 - Programação Orientada a Objetos
    \item \textbf{Universidade:} UFLA
\end{itemize}
\end{frame}

\end{document}

\documentclass[aspectratio=169]{beamer}

% Configurações do tema
\usetheme{Berkeley}
\usecolortheme{spruce}
\setbeamertemplate{navigation symbols}{}
\setbeamertemplate{footline}[frame number]

% Configuração do logo da UFLA
\setbeamertemplate{headline}{%
  \begin{beamercolorbox}[wd=\paperwidth,ht=1.2cm,dp=0.2cm]{section in head/foot}%
    \begin{center}
      \vspace{0.1cm}
      \includegraphics[height=1.0cm]{../../Image/ufla-logo.PNG}
    \end{center}
  \end{beamercolorbox}%
}

% Configuração do título centralizado
\setbeamertemplate{frametitle}{%
  \begin{beamercolorbox}[wd=\paperwidth,ht=1.2cm,dp=0.3cm]{frametitle}%
    \begin{center}
      \insertframetitle
    \end{center}
  \end{beamercolorbox}%
}

% Pacotes necessários
\usepackage[utf8]{inputenc}
\usepackage[T1]{fontenc}
\usepackage[brazilian]{babel}
\usepackage{amsmath}
\usepackage{amsfonts}
\usepackage{amssymb}
\usepackage{graphicx}
\usepackage{hyperref}
\usepackage{multicol}
\usepackage{listings}
\usepackage{xcolor}
\usepackage{textcomp}
\usepackage{verbatim}

% Configurações de cores
\definecolor{azul}{RGB}{0,51,102}
\definecolor{azulclaro}{RGB}{51,102,153}
\definecolor{lightblue}{RGB}{173,216,230}
\definecolor{lightgreen}{RGB}{144,238,144}
\definecolor{lightcoral}{RGB}{240,128,128}
\definecolor{lightyellow}{RGB}{255,255,224}
\definecolor{lightpink}{RGB}{255,182,193}
\definecolor{lightgray}{RGB}{211,211,211}
\definecolor{orange}{RGB}{255,165,0}

% Configuração de cores do tema
\setbeamercolor{title}{fg=white,bg=azul}
\setbeamercolor{frametitle}{fg=white,bg=azulclaro}
\setbeamercolor{section title}{fg=white,bg=azul}

% Configuração do listings para C#
\lstset{
    language=[Sharp]C,
    basicstyle=\scriptsize\ttfamily,
    keywordstyle=\color{blue},
    commentstyle=\color{green!60!black},
    stringstyle=\color{red},
    numbers=left,
    numberstyle=\tiny,
    stepnumber=1,
    numbersep=5pt,
    backgroundcolor=\color{lightgray},
    showspaces=false,
    showstringspaces=false,
    frame=single,
    tabsize=2,
    breaklines=true,
    extendedchars=true,
    literate={á}{{\'a}}1 {à}{{\`a}}1 {ã}{{\~a}}1 {â}{{\^a}}1 {é}{{\'e}}1 {ê}{{\^e}}1 {í}{{\'i}}1 {ó}{{\'o}}1 {ô}{{\^o}}1 {õ}{{\~o}}1 {ú}{{\'u}}1 {ç}{{\c c}}1
}

% Informações do documento
\title{Programação Orientada a Objetos}
\subtitle{GCT052 - Aula 3.1 - Relações entre Classes}
\author{Prof. Dr. Bento Rafael Siqueira}
\institute{Universidade Federal de Lavras\\Departamento de Ciência da Computação}
\date{\today}

\begin{document}

% Slide de título
\begin{frame}
\titlepage
\end{frame}

% Slide de sumário
\begin{frame}{Sumário}
\tableofcontents
\end{frame}

% Seção 1: Introdução
\section{Introdução às Relações}

\begin{frame}{Por que Relações entre Classes?}
\begin{itemize}
\item Classes não existem isoladas no mundo real
\item Sistemas complexos requerem \textbf{interação} entre componentes
\item As relações definem \textbf{como} os objetos se relacionam
\item Importante para modelar sistemas reais corretamente
\end{itemize}

\vspace{0.5cm}
\begin{block}{Principais Tipos de Relação}
\begin{enumerate}
\item \textbf{Associação}: Relação genérica entre classes
\item \textbf{Agregação}: Relação "tem um" onde a parte pode existir sem o todo
\item \textbf{Composição}: Relação "tem um" onde a parte não existe sem o todo
\end{enumerate}
\end{block}
\end{frame}

\section{Associação}

\begin{frame}{Associação}
\begin{block}{Definição}
Relacionamento genérico entre duas classes independentes, onde uma classe usa outra.
\end{block}

\begin{itemize}
\item Classe A usa ou conhece Classe B
\item São \textbf{independentes}: existem separadamente
\item Sem relação de "propriedade"
\item A relação é \textbf{bidirecional} ou \textbf{unidirecional}
\end{itemize}

\vspace{0.5cm}
\begin{exampleblock}{Exemplo do Mundo Real}
\begin{itemize}
\item \texttt{Professor} $\leftrightarrow$ \texttt{Aluno} (bidirecional)
\item \texttt{Pessoa} usa \texttt{Telefone} (unidirecional)
\end{itemize}
\end{exampleblock}
\end{frame}

\begin{frame}{Associação - Exemplo em C\#}
\begin{lstlisting}
// Classe Professor
public class Professor
{
    public string Nome { get; set; }
    public List<Aluno> Alunos { get; set; }
    
    public Professor(string nome)
    {
        Nome = nome;
        Alunos = new List<Aluno>();
    }
    
    public void AdicionarAluno(Aluno aluno)
    {
        Alunos.Add(aluno);
    }
}

// Classe Aluno
public class Aluno
{
    public string Nome { get; set; }
    public string Matricula { get; set; }
    
    public Aluno(string nome, string matricula)
    {
        Nome = nome;
        Matricula = matricula;
    }
}
\end{lstlisting}
\end{frame}

\begin{frame}[fragile]{Associação - Notação UML}
\begin{center}
\begin{verbatim}
+-------------+
|  Professor  |
|  - Nome     |
|  - Alunos   |
+------+------+
       | *
       |
       v
+-------------+
|   Aluno     |
|   - Nome    |
|   - Matricula
+-------------+
\end{verbatim}
\end{center}

\vspace{0.5cm}
\begin{block}{Características}
\begin{itemize}
\item Relação simples entre duas classes
\item Professor pode ter nenhum, um ou muitos alunos
\item Aluno pode existir sem Professor
\end{itemize}
\end{block}
\end{frame}

\section{Agregação}

\begin{frame}{Agregação}
\begin{block}{Definição}
Relação "tem um" onde a parte pode existir independentemente do todo.
\end{block}

\vspace{0.5cm}
\begin{itemize}
\item Representa um relacionamento \textbf{mais forte} que associação
\item Relação de "propriedade parcial"
\item A parte \textbf{pode sobreviver} sem o todo
\item Destruição do todo \textbf{não destrói} as partes
\end{itemize}

\vspace{0.5cm}
\begin{exampleblock}{Exemplo do Mundo Real}
\begin{itemize}
\item \texttt{Universidade} "tem" \texttt{Departamentos}
\item \texttt{Biblioteca} "tem" \texttt{Livros}
\item Se a universidade for fechada, os departamentos continuam existindo
\end{itemize}
\end{exampleblock}
\end{frame}

\begin{frame}{Agregação - Exemplo em C\#}
\begin{lstlisting}
// Classe Departamento (parte)
public class Departamento
{
    public string Nome { get; set; }
    public string Sigla { get; set; }
    
    public Departamento(string nome, string sigla)
    {
        Nome = nome;
        Sigla = sigla;
    }
}

// Classe Universidade (todo)
public class Universidade
{
    public string Nome { get; set; }
    public List<Departamento> Departamentos { get; set; }
    
    public Universidade(string nome)
    {
        Nome = nome;
        Departamentos = new List<Departamento>();
    }
    
    public void AdicionarDepartamento(Departamento dept)
    {
        Departamentos.Add(dept);
    }
}
\end{lstlisting}
\end{frame}

\begin{frame}[fragile]{Agregação - Notação UML}
\begin{center}
\begin{verbatim}
+------------------+
|   Universidade   |
|   - Nome         |
|   - Departamentos|
+----------+-------+
           | (diamond)
           |
           v
+------------------+
|  Departamento    |
|  - Nome          |
|  - Sigla         |
+------------------+
\end{verbatim}
\end{center}

\vspace{0.5cm}
\begin{block}{Características}
\begin{itemize}
\item Representada por um \textbf{losango vazio} ($\lozenge$)
\item Universidade "tem" Departamentos
\item Departamento pode existir sem Universidade
\end{itemize}
\end{block}
\end{frame}

\section{Composição}

\begin{frame}{Composição}
\begin{block}{Definição}
Relação "tem um" onde a parte \textbf{não pode existir} sem o todo.
\end{block}

\vspace{0.5cm}
\begin{itemize}
\item Relação \textbf{mais forte} que agregação
\item Relação de "propriedade completa"
\item A parte \textbf{não sobrevive} sem o todo
\item Destruição do todo \textbf{destrói} as partes
\end{itemize}

\vspace{0.5cm}
\begin{exampleblock}{Exemplo do Mundo Real}
\begin{itemize}
\item \texttt{Casa} "tem" \texttt{Salas} - Se a casa for demolida, as salas também são destruídas
\item \texttt{ContaBancaria} "tem" \texttt{Transacoes} - Sem a conta, as transações não fazem sentido
\item \texttt{Arquivo} "tem" \texttt{Linhas} - Sem o arquivo, as linhas não existem
\end{itemize}
\end{exampleblock}
\end{frame}

\begin{frame}{Composição - Exemplo em C\#}
\begin{lstlisting}
// Classe Transacao (parte)
public class Transacao
{
    public DateTime Data { get; set; }
    public double Valor { get; set; }
    public string Tipo { get; set; }
    
    public Transacao(DateTime data, double valor, string tipo)
    {
        Data = data;
        Valor = valor;
        Tipo = tipo;
    }
}

// Classe ContaBancaria (todo)
public class ContaBancaria
{
    public string Numero { get; set; }
    public double Saldo { get; set; }
    public List<Transacao> Transacoes { get; set; }
    
    public ContaBancaria(string numero)
    {
        Numero = numero;
        Saldo = 0;
        Transacoes = new List<Transacao>();
    }
    
    public void AdicionarTransacao(Transacao trans)
    {
        Transacoes.Add(trans);
        Saldo += trans.Valor;
    }
}
\end{lstlisting}
\end{frame}

\begin{frame}[fragile]{Composição - Notação UML}
\begin{center}
\begin{verbatim}
+------------------+
| ContaBancaria    |
|  - Numero        |
|  - Saldo         |
|  - Transacoes    |
+----------+-------+
           | (filled diamond)
           |
           v
+------------------+
|   Transacao      |
|   - Data         |
|   - Valor        |
|   - Tipo         |
+------------------+
\end{verbatim}
\end{center}

\vspace{0.5cm}
\begin{block}{Características}
\begin{itemize}
\item Representada por um \textbf{losango preenchido} ($\blacklozenge$)
\item ContaBancaria "tem" Transacoes
\item Transacao \textbf{não pode existir} sem ContaBancaria
\end{itemize}
\end{block}
\end{frame}

\section{Comparação}

\begin{frame}{Comparação das Relações}
\begin{table}
\centering
\small
\begin{tabular}{|l|p{3cm}|p{3cm}|p{3cm}|}
\hline
\textbf{Aspecto} & \textbf{Associação} & \textbf{Agregação} & \textbf{Composição} \\
\hline
Relação & Usa/Conhece & "Tem um" & "Tem um" \\
\hline
Dependência & Independe & Leve & Forte \\
\hline
Existência & Totalmente independente & Parte pode existir sem & Parte não existe sem \\
\hline
Lifecycle & Separado & Separado & Controlado pelo todo \\
\hline
UML & Linha simples & Losango vazio & Losango preenchido \\
\hline
Exemplo & Professor-Aluno & Universidade-Dept & Casa-Sala \\
\hline
\end{tabular}
\end{table}

\vspace{0.3cm}
\begin{alertblock}{Regra de Ouro}
\begin{itemize}
\item \textbf{Associação}: "usa"
\item \textbf{Agregação}: "tem parcialmente"
\item \textbf{Composição}: "tem totalmente"
\end{itemize}
\end{alertblock}
\end{frame}

\begin{frame}{Decidindo o Tipo de Relação}
\begin{block}{Perguntas para Decidir}
\end{block}

\begin{enumerate}
\item \textbf{As classes podem existir independentemente?}
\begin{itemize}
\item Sim $\rightarrow$ Associação
\item Não $\rightarrow$ Composição ou Agregação
\end{itemize}

\vspace{0.3cm}
\item \textbf{A parte tem sentido sem o todo?}
\begin{itemize}
\item Sim $\rightarrow$ Agregação
\item Não $\rightarrow$ Composição
\end{itemize}

\vspace{0.3cm}
\item \textbf{O todo controla a existência da parte?}
\begin{itemize}
\item Sim $\rightarrow$ Composição
\item Não $\rightarrow$ Agregação
\end{itemize}
\end{enumerate}
\end{frame}

\section{Exemplos Práticos}

\begin{frame}{Exemplo Completo 1: Biblioteca}
\begin{block}{Sistema de Biblioteca}
\begin{itemize}
\item \texttt{Biblioteca} - Classe principal
\item \texttt{Livro} - Pode existir sem biblioteca? \textbf{Sim} $\rightarrow$ Agregação
\item \texttt{Pessoa} - Usa a biblioteca? \textbf{Sim} $\rightarrow$ Associação
\item \texttt{Emprestimo} - Existe sem biblioteca? \textbf{Não} $\rightarrow$ Composição
\end{itemize}
\end{block}

\vspace{0.5cm}
\begin{lstlisting}[basicstyle=\tiny]
public class Biblioteca
{
    private List<Livro> livros;      // Agregação
    private List<Pessoa> pessoas;    // Associação
    private List<Emprestimo> emprestimos; // Composição
}

public class Livro { }               // Existe independente

public class Pessoa { }              // Existe independente

public class Emprestimo { }          // Não existe sem Biblioteca
\end{lstlisting}
\end{frame}

\begin{frame}{Exemplo Completo 2: Sistema Bancário}
\begin{block}{Sistema Bancário}
\begin{itemize}
\item \texttt{Banco} - Classe principal
\item \texttt{Cliente} - Relação? \textbf{Associação}
\item \texttt{Conta} - Relação? \textbf{Agregação} (conta pode fechar mas cliente existe)
\item \texttt{Extrato} - Relação? \textbf{Composição} (extrato não existe sem conta)
\item \texttt{Cartao} - Relação? \textbf{Composição} (cartão não existe sem conta)
\end{itemize}
\end{block}

\vspace{0.5cm}
\begin{lstlisting}[basicstyle=\tiny]
public class Banco
{
    private List<Cliente> clientes;     // Associação
    private List<Conta> contas;        // Agregação
    
    public void AdicionarCliente(Cliente cliente) { }
}

public class Conta
{
    private List<Transacao> transacoes; // Composição
    private Cartao cartao;              // Composição
}
\end{lstlisting}
\end{frame}

\section{Boas Práticas}

\begin{frame}{Boas Práticas no Uso de Relações}
\begin{block}{1. Composição para Dados Críticos}
\begin{itemize}
\item Use composição para dados que devem ser sempre consistentes
\item Exemplo: Transações de uma conta bancária
\end{itemize}
\end{block}

\begin{block}{2. Agregação para Compartilhamento}
\begin{itemize}
\item Use agregação quando vários objetos podem compartilhar partes
\item Exemplo: Vários departamentos podem ter o mesmo professor
\end{itemize}
\end{block}

\begin{block}{3. Associação para Colaboração}
\begin{itemize}
\item Use associação quando classes trabalham juntas mas são independentes
\item Exemplo: Professor e Aluno
\end{itemize}
\end{block}
\end{frame}

\begin{frame}{Armadilhas Comuns}
\begin{alertblock}{Erro 1: Usar Composição no Lugar de Agregação}
\begin{itemize}
\item \textcolor{red}{Errado}: Pensar que toda relação "tem um" é composição
\item \textcolor{green!70!black}{Certo}: Avalie se a parte sobrevive ao todo
\end{itemize}
\end{alertblock}

\begin{alertblock}{Erro 2: Confundir Associação com Agregação}
\begin{itemize}
\item \textcolor{red}{Errado}: Toda vez que há uma lista de objetos
\item \textcolor{green!70!black}{Certo}: Agregação implica relação de "propriedade"
\end{itemize}
\end{alertblock}

\begin{alertblock}{Erro 3: Não Considerar o Lifecycle}
\begin{itemize}
\item \textcolor{red}{Errado}: Apenas olhar o código, não o significado
\item \textcolor{green!70!black}{Certo}: Pense no ciclo de vida dos objetos
\end{itemize}
\end{alertblock}
\end{frame}

\section{Exercícios}

\begin{frame}{Exercício Proposto}
\begin{block}{Sistema de Restaurante}
Desenhe as classes e defina os tipos de relacionamento para:

\begin{itemize}
\item \texttt{Restaurante} - O estabelecimento
\item \texttt{Garcon} - Pessoal do restaurante
\item \texttt{Cliente} - Quem come no restaurante
\item \texttt{Cardapio} - Menu disponível
\item \texttt{Prato} - Item do cardápio
\item \texttt{Pedido} - Ordem feita pelo cliente
\item \texttt{ItemPedido} - Item específico em um pedido
\end{itemize}
\end{block}

\vspace{0.5cm}
\begin{exampleblock}{Para Pensar}
\begin{itemize}
\item Cardápio pode existir sem restaurante?
\item ItemPedido faz sentido sem Pedido?
\item Cliente e Garcon têm que existir juntos?
\end{itemize}
\end{exampleblock}
\end{frame}

\section{Conclusão}

\begin{frame}{Resumo}
\begin{block}{Associação}
\begin{itemize}
\item Relação mais fraca
\item Classes independentes
\item Representação: Linha simples
\end{itemize}
\end{block}

\begin{block}{Agregação}
\begin{itemize}
\item Relação intermediária
\item Parte pode existir sem o todo
\item Representação: Losango vazio
\end{itemize}
\end{block}

\begin{block}{Composição}
\begin{itemize}
\item Relação mais forte
\item Parte não existe sem o todo
\item Representação: Losango preenchido
\end{itemize}
\end{block}
\end{frame}

\begin{frame}{Fim}
\centering
\Large \textbf{Qualquer Dúvida?}

\vspace{1cm}

\normalsize
\textbf{Contato:} bento.siqueira@ufla.br

\vspace{0.5cm}

\textbf{Material:} github.com/californi/disciplinas-ufla
\end{frame}

\end{document}


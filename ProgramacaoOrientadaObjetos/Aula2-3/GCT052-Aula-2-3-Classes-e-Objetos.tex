\documentclass[aspectratio=169]{beamer}

% Configurações do tema
\usetheme{Berkeley}
\usecolortheme{spruce}
\setbeamertemplate{navigation symbols}{}
\setbeamertemplate{footline}[frame number]

% Configuração do logo da UFLA
\setbeamertemplate{headline}{%
  \begin{beamercolorbox}[wd=\paperwidth,ht=1.2cm,dp=0.2cm]{section in head/foot}%
    \begin{center}
      \vspace{0.1cm}
      \includegraphics[height=1.0cm]{../../Image/ufla-logo.PNG}
    \end{center}
  \end{beamercolorbox}%
}

% Configuração do título centralizado
\setbeamertemplate{frametitle}{%
  \begin{beamercolorbox}[wd=\paperwidth,ht=1.2cm,dp=0.3cm]{frametitle}%
    \begin{center}
      \insertframetitle
    \end{center}
  \end{beamercolorbox}%
}

% Pacotes necessários
\usepackage[utf8]{inputenc}
\usepackage[brazilian]{babel}
\usepackage{amsmath}
\usepackage{amsfonts}
\usepackage{amssymb}
\usepackage{graphicx}
\usepackage{hyperref}
\usepackage{multicol}
% Configurações de cores
\definecolor{azul}{RGB}{0,51,102}
\definecolor{azulclaro}{RGB}{51,102,153}
\definecolor{lightblue}{RGB}{173,216,230}
\definecolor{lightgreen}{RGB}{144,238,144}
\definecolor{lightcoral}{RGB}{240,128,128}
\definecolor{lightyellow}{RGB}{255,255,224}
\definecolor{lightpink}{RGB}{255,182,193}
\definecolor{lightgray}{RGB}{211,211,211}
\definecolor{orange}{RGB}{255,165,0}
\setbeamercolor{title}{fg=white,bg=azul}
\setbeamercolor{frametitle}{fg=white,bg=azulclaro}
\setbeamercolor{section title}{fg=white,bg=azul}


% Informações do documento
\title{Programação Orientada a Objetos}
\subtitle{GCT052 - Aula 2.3 - Classes e Objetos}
\author{Prof. Dr. Bento Rafael Siqueira}
\institute{Universidade Federal de Lavras\\Departamento de Ciência da Computação}
\date{\today}

\begin{document}

% Slide de título
\begin{frame}
\titlepage
\end{frame}

% Slide de sumário
\begin{frame}{Sumário}
\tableofcontents
\end{frame}

% Seção 1: Introdução
\section{Introdução às Classes e Objetos}

\begin{frame}{O que são Classes e Objetos?}
\begin{itemize}
\item \textbf{Classe}: É um molde ou template que define as características e comportamentos de um tipo de objeto
\item \textbf{Objeto}: É uma instância específica de uma classe, com valores únicos para seus atributos
\item \textbf{Analogia}: Classe = Planta de uma casa, Objeto = Casa construída
\end{itemize}

\vspace{0.5cm}
\textbf{Exemplo do Dia a Dia:}
\begin{itemize}
\item \textbf{Classe}: Carro
\item \textbf{Atributos}: Marca, modelo, cor, ano, placa
\item \textbf{Métodos}: Ligar, acelerar, frear, parar
\item \textbf{Objetos}: Meu carro (Fiat, Palio, branco, 2020, ABC-1234)
\end{itemize}
\end{frame}

\begin{frame}{Por que usar Classes e Objetos?}
\begin{enumerate}
\item \textbf{Organização}: Código mais estruturado e fácil de entender
\item \textbf{Reutilização}: Uma classe pode gerar múltiplos objetos
\item \textbf{Manutenção}: Mudanças na classe afetam todos os objetos
\item \textbf{Abstração}: Foco no que é importante, escondendo detalhes
\item \textbf{Encapsulamento}: Proteção dos dados internos
\end{enumerate}

\vspace{0.5cm}
\textbf{Vantagens:}
\begin{itemize}
\item Código mais limpo e legível
\item Facilita trabalho em equipe
\item Reduz duplicação de código
\item Permite modelagem do mundo real
\end{itemize}
\end{frame}

% Seção 2: Conceitos Fundamentais
\section{Conceitos Fundamentais}

\begin{frame}{Estrutura de uma Classe}
\begin{itemize}
\item \textbf{Atributos (Campos)}: Características ou propriedades da classe
\item \textbf{Métodos}: Ações ou comportamentos que a classe pode realizar
\item \textbf{Construtor}: Método especial para criar objetos
\item \textbf{Modificadores de Acesso}: Controlam quem pode acessar os membros
\end{itemize}

\vspace{0.3cm}
\textbf{Exemplo Básico:}
\begin{itemize}
\item \textbf{Classe}: Pessoa
\item \textbf{Atributos}: Nome, idade, email
\item \textbf{Métodos}: Falar, andar, comer
\item \textbf{Construtor}: Criar uma nova pessoa
\end{itemize}
\end{frame}

\begin{frame}{Modificadores de Acesso}
\begin{enumerate}
\item \textbf{public}: Acessível de qualquer lugar
\item \textbf{private}: Acessível apenas dentro da própria classe
\item \textbf{protected}: Acessível na classe e nas classes filhas
\item \textbf{internal}: Acessível apenas dentro do mesmo assembly
\end{enumerate}

\vspace{0.5cm}
\textbf{Boas Práticas:}
\begin{itemize}
\item Atributos geralmente são \textbf{private}
\item Métodos públicos para interação externa
\item Use \textbf{properties} para acesso controlado aos atributos
\end{itemize}
\end{frame}

% Seção 3: Exemplos Práticos
\section{Exemplos Práticos}

\begin{frame}{Exemplo 1: Classe Conta Bancária - Estrutura}
\textbf{Atributos:}
\begin{itemize}
\item Número da conta
\item Nome do titular
\item Saldo
\item Tipo da conta
\end{itemize}

\vspace{0.5cm}
\textbf{Métodos:}
\begin{itemize}
\item Depositar
\item Sacar
\item Verificar saldo
\item Transferir
\end{itemize}
\end{frame}

\begin{frame}{Exemplo 1: Classe Conta Bancária - Objetos}
\textbf{Objetos possíveis:}
\begin{itemize}
\item Conta de João (12345, João Silva, R\$ 1.500,00, Corrente)
\item Conta de Maria (67890, Maria Santos, R\$ 3.200,00, Poupança)
\end{itemize}

\vspace{0.5cm}
\textbf{Características dos Objetos:}
\begin{itemize}
\item Cada objeto tem seus próprios valores
\item Podem executar os mesmos métodos
\item Estado independente entre objetos
\end{itemize}
\end{frame}

\begin{frame}{Exemplo 2: Classe Animal - Estrutura}
\textbf{Atributos:}
\begin{itemize}
\item Nome
\item Espécie
\item Idade
\item Peso
\item Cor
\end{itemize}

\vspace{0.5cm}
\textbf{Métodos:}
\begin{itemize}
\item Comer
\item Dormir
\item Mover
\item Fazer som
\end{itemize}
\end{frame}

\begin{frame}{Exemplo 2: Classe Animal - Objetos}
\textbf{Objetos possíveis:}
\begin{itemize}
\item Meu gato (Mimi, Gato, 3 anos, 4kg, Branco)
\item Cachorro do vizinho (Rex, Cachorro, 5 anos, 15kg, Marrom)
\end{itemize}

\vspace{0.5cm}
\textbf{Comportamentos dos Objetos:}
\begin{itemize}
\item Cada animal tem características únicas
\item Comportamentos podem variar por espécie
\item Estado muda com ações (comer, dormir)
\end{itemize}
\end{frame}

\begin{frame}{Exemplo 3: Classe Produto - Estrutura}
\textbf{Atributos:}
\begin{itemize}
\item Código
\item Nome
\item Preço
\item Quantidade em estoque
\item Categoria
\end{itemize}

\vspace{0.5cm}
\textbf{Métodos:}
\begin{itemize}
\item Calcular desconto
\item Atualizar estoque
\item Verificar disponibilidade
\item Aplicar promoção
\end{itemize}
\end{frame}

\begin{frame}{Exemplo 3: Classe Produto - Objetos}
\textbf{Objetos possíveis:}
\begin{itemize}
\item Notebook (NB001, Dell Inspiron, R\$ 2.500,00, 10, Eletrônicos)
\item Livro (LB002, C\# Completo, R\$ 89,90, 25, Livros)
\end{itemize}

\vspace{0.5cm}
\textbf{Funcionalidades dos Objetos:}
\begin{itemize}
\item Controle de estoque
\item Cálculo de preços
\item Gestão de categorias
\item Aplicação de descontos
\end{itemize}
\end{frame}

% Seção 4: Implementação em C#
\section{Implementação em C\#}

\begin{frame}{Estrutura Básica de uma Classe em C\#}
\textbf{Elementos principais:}
\begin{itemize}
\item \textbf{Atributos}: \texttt{private string nome; private int idade;}
\item \textbf{Construtor}: \texttt{public NomeDaClasse(string nome, int idade)}
\item \textbf{Métodos}: \texttt{public void Falar()}
\item \textbf{Properties}: \texttt{public string Nome \{ get; set; \}}
\end{itemize}

\vspace{0.3cm}
\textbf{Exemplo completo:} Ver arquivo \texttt{Exemplos/ContaBancaria.cs}
\end{frame}

\begin{frame}{Criando e Usando Objetos}
\textbf{Passos para usar objetos:}
\begin{enumerate}
\item \textbf{Criar objetos}: \texttt{NomeDaClasse obj = new NomeDaClasse();}
\item \textbf{Usar métodos}: \texttt{obj.Falar(); obj.Comer();}
\item \textbf{Acessar properties}: \texttt{obj.Nome = "João";}
\item \textbf{Cada objeto é independente}: Valores únicos para cada instância
\end{enumerate}

\vspace{0.3cm}
\textbf{Exemplo prático:} Ver arquivo \texttt{Exemplos/Program.cs}
\end{frame}

% Seção 5: Exemplos de Código
\section{Exemplos de Código}

\begin{frame}{Exemplo Completo: Classe Conta Bancária}
\textbf{Arquivo: Exemplos/ContaBancaria.cs}
\begin{itemize}
\item Implementação completa da classe ContaBancaria
\item Atributos: numero, titular, saldo, tipo
\item Métodos: Depositar, Sacar, VerificarSaldo, Transferir
\item Construtor para inicializar a conta
\item Properties para acesso controlado
\end{itemize}

\vspace{0.3cm}
\textbf{Características:}
\begin{itemize}
\item Validação de valores negativos
\item Controle de saldo insuficiente
\item Mensagens informativas
\item Código bem documentado
\end{itemize}
\end{frame}

\begin{frame}{Exemplo Completo: Classe Animal}
\textbf{Arquivo: Exemplos/Animal.cs}
\begin{itemize}
\item Implementação da classe Animal
\item Atributos: nome, especie, idade, peso, cor
\item Métodos: Comer, Dormir, Mover, FazerSom
\item Construtor com parâmetros opcionais
\item Método ToString() para exibição
\end{itemize}

\vspace{0.3cm}
\textbf{Características:}
\begin{itemize}
\item Validação de dados de entrada
\item Comportamentos específicos por espécie
\item Cálculo de idade em meses
\item Interface amigável
\end{itemize}
\end{frame}

\begin{frame}{Exemplo Completo: Classe Produto}
\textbf{Arquivo: Exemplos/Produto.cs}
\begin{itemize}
\item Implementação da classe Produto
\item Atributos: codigo, nome, preco, estoque, categoria
\item Métodos: CalcularDesconto, AtualizarEstoque, VerificarDisponibilidade
\item Construtor com validações
\item Properties com validação
\end{itemize}

\vspace{0.3cm}
\textbf{Características:}
\begin{itemize}
\item Cálculo automático de descontos
\item Controle de estoque
\item Validação de preços
\item Categorização de produtos
\end{itemize}
\end{frame}

% Seção 6: Exercícios Práticos
\section{Exercícios Práticos}

\begin{frame}{Exercício 1: Classe Livro}
\textbf{Crie uma classe Livro com:}

\textbf{Atributos:}
\begin{itemize}
\item Título
\item Autor
\item ISBN
\item Páginas
\item Preço
\item Ano de publicação
\end{itemize}

\textbf{Métodos:}
\begin{itemize}
\item Calcular preço com desconto
\item Verificar se é livro novo (últimos 2 anos)
\item Exibir informações do livro
\item Calcular preço por página
\end{itemize}

\vspace{0.3cm}
\textbf{Tempo}: 15 minutos
\end{frame}

\begin{frame}{Exercício 2: Classe Retângulo}
\textbf{Crie uma classe Retângulo com:}

\textbf{Atributos:}
\begin{itemize}
\item Largura
\item Altura
\end{itemize}

\textbf{Métodos:}
\begin{itemize}
\item Calcular área
\item Calcular perímetro
\item Verificar se é quadrado
\item Desenhar retângulo (usando asteriscos)
\end{itemize}

\vspace{0.3cm}
\textbf{Tempo}: 10 minutos
\end{frame}

\begin{frame}{Exercício 3: Classe Estudante - Estrutura}
\textbf{Crie uma classe Estudante com:}

\textbf{Atributos:}
\begin{itemize}
\item Nome
\item Matrícula
\item Curso
\item Notas (array de 4 notas)
\end{itemize}

\vspace{0.5cm}
\textbf{Métodos:}
\begin{itemize}
\item Calcular média
\item Verificar se passou (média >= 7.0)
\item Adicionar nota
\item Exibir boletim
\end{itemize}
\end{frame}

\begin{frame}{Exercício 3: Classe Estudante - Implementação}
\textbf{Tempo}: 20 minutos

\vspace{0.5cm}
\textbf{Dicas para implementação:}
\begin{itemize}
\item Use arrays para armazenar notas
\item Implemente validação de notas (0-10)
\item Calcule média aritmética simples
\item Formate a saída do boletim
\end{itemize}

\vspace{0.3cm}
\textbf{Teste com diferentes estudantes!}
\end{frame}

% Seção 7: Boas Práticas
\section{Boas Práticas}

\begin{frame}{Nomenclatura}
\begin{itemize}
\item \textbf{Classes}: PascalCase (ex: ContaBancaria, Produto)
\item \textbf{Métodos}: PascalCase (ex: Depositar, CalcularSaldo)
\item \textbf{Atributos}: camelCase (ex: nome, saldoAtual)
\item \textbf{Constantes}: UPPER\_CASE (ex: TAXA\_JUROS)
\end{itemize}

\vspace{0.5cm}
\textbf{Exemplos:}
\begin{itemize}
\item \textbf{Bom}: \texttt{public class ContaBancaria}
\item \textbf{Ruim}: \texttt{public class contabancaria}
\item \textbf{Bom}: \texttt{private string nomeTitular}
\item \textbf{Ruim}: \texttt{private string NomeTitular}
\end{itemize}
\end{frame}

\begin{frame}{Encapsulamento}
\begin{itemize}
\item \textbf{Sempre} use modificadores de acesso apropriados
\item Atributos devem ser \textbf{private}
\item Use \textbf{properties} para acesso controlado
\item Evite atributos \textbf{public}
\end{itemize}

\vspace{0.5cm}
\textbf{Exemplo:}
\begin{itemize}
\item \textbf{Bom}: \texttt{private double saldo;} + \texttt{public double Saldo \{ get; set; \}}
\item \textbf{Ruim}: \texttt{public double saldo;}
\end{itemize}

\vspace{0.3cm}
\textbf{Vantagens:}
\begin{itemize}
\item Controle de acesso aos dados
\item Validação automática
\item Facilita manutenção
\item Melhora segurança
\end{itemize}
\end{frame}

\begin{frame}{Documentação e Comentários - Boas Práticas}
\begin{itemize}
\item Use \textbf{XML comments} para documentar classes e métodos
\item Comente lógica complexa
\item Use nomes descritivos (evite comentários óbvios)
\item Documente parâmetros e valores de retorno
\end{itemize}

\vspace{0.5cm}
\textbf{Benefícios:}
\begin{itemize}
\item Facilita manutenção do código
\item Melhora compreensão para outros desenvolvedores
\item Gera documentação automática
\end{itemize}
\end{frame}

\begin{frame}{Documentação e Comentários - Exemplo}
\textbf{Exemplo de XML Comments:}
\begin{itemize}
\item \texttt{/// <summary>}
\item \texttt{/// Deposita um valor na conta}
\item \texttt{/// </summary>}
\item \texttt{/// <param name="valor">Valor a ser depositado</param>}
\item \texttt{/// <returns>True se o depósito foi realizado</returns>}
\end{itemize}

\vspace{0.5cm}
\textbf{Resultado:}
\begin{itemize}
\item IntelliSense mostra a documentação
\item Facilita uso da classe por outros desenvolvedores
\end{itemize}
\end{frame}

% Seção 8: Próximos Passos
\section{Próximos Passos}

\begin{frame}{Tópicos Relacionados}
\begin{enumerate}
\item \textbf{Herança}
\begin{itemize}
\item Classes pai e filhas
\item Reutilização de código
\item Polimorfismo
\end{itemize}

\item \textbf{Interfaces}
\begin{itemize}
\item Contratos de comportamento
\item Múltiplas implementações
\item Abstração avançada
\end{itemize}

\item \textbf{Polimorfismo}
\begin{itemize}
\item Múltiplas formas
\item Sobrescrita de métodos
\item Ligação tardia
\end{itemize}
\end{enumerate}
\end{frame}

\begin{frame}{Leitura Recomendada}
\begin{itemize}
\item \textbf{Microsoft Docs}: Classes and Objects in C\#
\item \textbf{Clean Code}: Robert C. Martin
\item \textbf{Effective C\#}: Bill Wagner
\item \textbf{Pro C\# 10}: Andrew Troelsen
\end{itemize}

\vspace{0.5cm}
\textbf{Prática Recomendada:}
\begin{enumerate}
\item Implemente os exemplos da pasta \texttt{Exemplos}
\item Resolva os exercícios propostos
\item Crie suas próprias classes
\item Pratique com projetos reais
\end{enumerate}
\end{frame}

% Slide final
\begin{frame}{Dúvidas e Discussão}
\begin{center}
\textbf{Obrigado pela atenção!}
\end{center}

\textbf{Pontos-Chave da Aula}:
\begin{enumerate}
\item \textbf{Classes} são moldes que definem características e comportamentos
\item \textbf{Objetos} são instâncias específicas de classes
\item \textbf{Encapsulamento} protege os dados e controla o acesso
\item \textbf{Boas práticas} facilitam manutenção e legibilidade
\end{enumerate}

\vspace{0.5cm}
\begin{center}
\includegraphics[width=0.4\textwidth]{../../Image/ufla-logo.PNG}
\end{center}

\vspace{0.3cm}
\textbf{Próxima Aula}: Herança e Polimorfismo

\vspace{0.3cm}
\textbf{Contato}: bento.siqueira@ufla.br\\
\textbf{Material}: Disponível no sistema acadêmico
\end{frame}

\end{document}

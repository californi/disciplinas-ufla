\documentclass[aspectratio=169]{beamer}
\usepackage[utf8]{inputenc}
\usepackage[portuguese]{babel}
\usepackage{graphicx}
\usepackage{hyperref}
\usepackage{listings}
\usepackage{xcolor}
\usepackage{amsmath}
\usepackage{amsfonts}
\usepackage{amssymb}
\usepackage{tikz}

% Configuração do tema
\usetheme{Madrid}
\usecolortheme{default}

% Configuração do código
\lstset{
    language=Python,
    basicstyle=\tiny\ttfamily,
    keywordstyle=\color{blue},
    commentstyle=\color{green!60!black},
    stringstyle=\color{red},
    numbers=left,
    numberstyle=\tiny,
    stepnumber=1,
    numbersep=5pt,
    backgroundcolor=\color{gray!10},
    showspaces=false,
    showstringspaces=false,
    showtabs=false,
    frame=single,
    tabsize=2,
    captionpos=b,
    breaklines=true,
    breakatwhitespace=false,
    escapeinside={\%*}{*)}}
}

% Informações do documento
\title[Minicurso REST API]{Minicurso: REST APIs com FastAPI, Microservices, Docker e Kubernetes}
\subtitle{Introdução Completa em 60 Minutos}
\author{Professor Dr. Bento Rafael Siqueira}
\institute{UFLA - Universidade Federal de Lavras}
\date{\today}

\begin{document}

% Slide de título
\begin{frame}
    \titlepage
\end{frame}

% Sumário
\begin{frame}
    \frametitle{Agenda}
    \tableofcontents
\end{frame}

\section{Introdução}

\begin{frame}
    \frametitle{O que vamos aprender hoje?}
    \begin{columns}
        \begin{column}{0.5\textwidth}
            \textbf{Conceitos Fundamentais:}
            \begin{itemize}
                \item REST APIs
                \item FastAPI Framework
                \item Microservices
                \item Containerização
                \item Orquestração
            \end{itemize}
        \end{column}
        \begin{column}{0.5\textwidth}
            \textbf{Tecnologias:}
            \begin{itemize}
                \item Python 3.8+
                \item FastAPI
                \item Docker
                \item Kubernetes
                \item Docker Compose
            \end{itemize}
        \end{column}
    \end{columns}
    
    \vspace{1em}
    \begin{alertblock}{Objetivo}
        Ao final desta aula, você será capaz de criar, containerizar e orquestrar uma API REST completa!
    \end{alertblock}
\end{frame}

\section{REST APIs}

\begin{frame}
    \frametitle{O que é uma REST API?}
    \begin{block}{REST - Representational State Transfer}
        \begin{itemize}
            \item Arquitetura de software para sistemas distribuídos
            \item Usa HTTP como protocolo de comunicação
            \item Baseada em recursos (Resources) identificados por URLs
            \item Operações através de métodos HTTP (GET, POST, PUT, DELETE)
        \end{itemize}
    \end{block}
    
    \begin{exampleblock}{Características Principais}
        \begin{itemize}
            \item \textbf{Stateless}: Cada requisição é independente
            \item \textbf{Client-Server}: Separação clara de responsabilidades
            \item \textbf{Cacheable}: Respostas podem ser armazenadas em cache
            \item \textbf{Uniform Interface}: Interface consistente
        \end{itemize}
    \end{exampleblock}
\end{frame}

\begin{frame}
    \frametitle{Métodos HTTP e seus significados}
    \begin{center}
        \begin{tabular}{|l|l|l|}
            \hline
            \textbf{Método} & \textbf{Operação} & \textbf{Descrição} \\
            \hline
            GET & Leitura & Recuperar dados \\
            \hline
            POST & Criação & Criar novo recurso \\
            \hline
            PUT & Atualização & Atualizar recurso completo \\
            \hline
            PATCH & Atualização & Atualizar parte do recurso \\
            \hline
            DELETE & Exclusão & Remover recurso \\
            \hline
        \end{tabular}
    \end{center}
    
    \vspace{1em}
    \begin{exampleblock}{Exemplo de URLs REST}
        \begin{itemize}
            \item \texttt{GET /api/users} - Listar todos os usuários
            \item \texttt{GET /api/users/123} - Obter usuário específico
            \item \texttt{POST /api/users} - Criar novo usuário
            \item \texttt{PUT /api/users/123} - Atualizar usuário
            \item \texttt{DELETE /api/users/123} - Deletar usuário
        \end{itemize}
    \end{exampleblock}
\end{frame}

\section{FastAPI}

\begin{frame}
    \frametitle{Por que FastAPI?}
    \begin{columns}
        \begin{column}{0.5\textwidth}
            \textbf{Vantagens:}
            \begin{itemize}
                \item \textbf{Performance}: Uma das APIs mais rápidas do Python
                \item \textbf{Documentação Automática}: Swagger/OpenAPI
                \item \textbf{<|tool▁calls▁begin|><|tool▁call▁begin|>
search_replace}type Hints}: Validação automática
                \item \textbf{Async/Await}: Suporte nativo
                \item \textbf{Fácil de usar}: Sintaxe simples
            \end{itemize}
        \end{column}
        \begin{column}{0.5\textwidth}
            \begin{alertblock}{Comparação de Performance}
                \begin{itemize}
                    \item FastAPI: ~60,000 req/s
                    \item Flask: ~20,000 req/s
                    \item Django: ~15,000 req/s
                \end{itemize}
            \end{alertblock}
        \end{column}
    \end{columns}
\end{frame}

\begin{frame}[fragile]
    \frametitle{Primeira API com FastAPI}
    \begin{lstlisting}[language=Python]
from fastapi import FastAPI
from pydantic import BaseModel
from typing import List

app = FastAPI(title="Minha API", version="1.0.0")

# Modelo de dados
class User(BaseModel):
    id: int
    name: str
    email: str

# Dados em memoria (para demonstracao)
users = [
    User(id=1, name="Joao", email="joao@email.com"),
    User(id=2, name="Maria", email="maria@email.com")
]

@app.get("/")
async def root():
    return {"message": "Hello World!"}

@app.get("/users", response_model=List[User])
async def get_users():
    return users

@app.get("/users/{user_id}", response_model=User)
async def get_user(user_id: int):
    for user in users:
        if user.id == user_id:
            return user
    return {"error": "User not found"}
    \end{lstlisting}
\end{frame}

\section{Microservices}

\begin{frame}
    \frametitle{Arquitetura de Microservices}
    \begin{columns}
        \begin{column}{0.5\textwidth}
            \textbf{Monolito vs Microservices}
            \begin{itemize}
                \item \textbf{Monolito}: Uma aplicação única
                \item \textbf{Microservices}: Múltiplas aplicações pequenas
            \end{itemize}
            
            \vspace{1em}
            \textbf{Vantagens dos Microservices:}
            \begin{itemize}
                \item Escalabilidade independente
                \item Tecnologias diferentes
                \item Deploy independente
                \item Falhas isoladas
            \end{itemize}
        \end{column}
        \begin{column}{0.5\textwidth}
            \begin{center}
                \begin{tikzpicture}[scale=0.8]
                    % Monolito
                    \draw[thick, fill=red!20] (0,0) rectangle (2,1.5);
                    \node at (1,0.75) {\small Monolito};
                    
                    % Microservices
                    \draw[thick, fill=green!20] (4,0) rectangle (5.5,1.5);
                    \draw[thick, fill=green!20] (6,0) rectangle (7.5,1.5);
                    \draw[thick, fill=green!20] (8,0) rectangle (9.5,1.5);
                    \draw[thick, fill=green!20] (10,0) rectangle (11.5,1.5);
                    
                    \node at (4.75,0.75) {\tiny User};
                    \node at (6.75,0.75) {\tiny Product};
                    \node at (8.75,0.75) {\tiny Order};
                    \node at (10.75,0.75) {\tiny Payment};
                    
                    % Seta
                    \draw[thick, ->] (2.2,0.75) -- (3.8,0.75);
                    \node at (3,1.2) {\tiny Evolucao};
                \end{tikzpicture}
            \end{center}
        \end{column}
    \end{columns}
\end{frame}

\begin{frame}
    \frametitle{Exemplo de Microservices}
    \begin{center}
        \begin{tabular}{|l|l|l|}
            \hline
            \textbf{Service} & \textbf{Porta} & \textbf{Responsabilidade} \\
            \hline
            User Service & 8001 & Gerenciar usuarios \\
            \hline
            Product Service & 8002 & Gerenciar produtos \\
            \hline
            Order Service & 8003 & Gerenciar pedidos \\
            \hline
            Payment Service & 8004 & Processar pagamentos \\
            \hline
            API Gateway & 8000 & Roteamento e autenticacao \\
            \hline
        \end{tabular}
    \end{center}
    
    \vspace{1em}
    \begin{exampleblock}{Comunicacao entre Services}
        \begin{itemize}
            \item \textbf{Sincrona}: HTTP/REST, gRPC
            \item \textbf{Assincrona}: Message Queues (RabbitMQ, Kafka)
            \item \textbf{Service Discovery}: Consul, Eureka
        \end{itemize}
    \end{exampleblock}
\end{frame}

\section{Docker}

\begin{frame}
    \frametitle{O que é Docker?}
    \begin{block}{Containerização}
        \begin{itemize}
            \item Empacota aplicação e suas dependências
            \item Ambiente isolado e portátil
            \item Consistência entre desenvolvimento e produção
            \item Escalabilidade horizontal
        \end{itemize}
    \end{block}
    
    \begin{columns}
        \begin{column}{0.5\textwidth}
            \textbf{Vantagens:}
            \begin{itemize}
                \item "Funciona na minha máquina"
                \item Deploy rápido
                \item Isolamento
                \item Recursos otimizados
            \end{itemize}
        \end{column}
        \begin{column}{0.5\textwidth}
            \textbf{Conceitos:}
            \begin{itemize}
                \item \textbf{Image}: Template
                \item \textbf{Container}: Instância
                \item \textbf{Dockerfile}: Receita
                \item \textbf{Registry}: Repositório
            \end{itemize}
        \end{column}
    \end{columns}
\end{frame}

\begin{frame}[fragile]
    \frametitle{Dockerfile para FastAPI}
    \begin{lstlisting}[language=bash]
# Usar imagem base do Python
FROM python:3.9-slim

# Definir diretorio de trabalho
WORKDIR /app

# Copiar arquivos de dependencias
COPY requirements.txt .

# Instalar dependencias
RUN pip install --no-cache-dir -r requirements.txt

# Copiar codigo da aplicacao
COPY . .

# Expor porta
EXPOSE 8000

# Comando para executar a aplicacao
CMD ["uvicorn", "main:app", "--host", "0.0.0.0", "--port", "8000"]
    \end{lstlisting}
    
    \begin{exampleblock}{requirements.txt}
        \begin{lstlisting}[language=bash]
fastapi==0.104.1
uvicorn[standard]==0.24.0
pydantic==2.5.0
        \end{lstlisting}
    \end{exampleblock}
\end{frame}

\begin{frame}[fragile]
    \frametitle{Docker Compose para Microservices}
    \begin{lstlisting}[language=bash]
version: '3.8'

services:
  user-service:
    build: ./user-service
    ports:
      - "8001:8000"
    environment:
      - DATABASE_URL=postgresql://user:pass@db:5432/users
    depends_on:
      - db

  product-service:
    build: ./product-service
    ports:
      - "8002:8000"
    environment:
      - DATABASE_URL=postgresql://user:pass@db:5432/products
    depends_on:
      - db

  api-gateway:
    build: ./api-gateway
    ports:
      - "8000:8000"
    depends_on:
      - user-service
      - product-service

  db:
    image: postgres:13
    environment:
      POSTGRES_DB: microservices
      POSTGRES_USER: user
      POSTGRES_PASSWORD: pass
    volumes:
      - postgres_data:/var/lib/postgresql/data

volumes:
  postgres_data:
    \end{lstlisting}
\end{frame}

\section{Kubernetes}

\begin{frame}
    \frametitle{O que é Kubernetes?}
    \begin{block}{Orquestração de Containers}
        \begin{itemize}
            \item Plataforma open-source para orquestração
            \item Automatiza deploy, scaling e gerenciamento
            \item Desenvolvido pelo Google
            \item Padrão da indústria
        \end{itemize}
    \end{block}
    
    \begin{columns}
        \begin{column}{0.5\textwidth}
            \textbf{Recursos Principais:}
            \begin{itemize}
                \item \textbf{Pods}: Unidade mínima
                \item \textbf{Services}: Descoberta e load balancing
                \item \textbf{Deployments}: Gerenciamento de replicas
                \item \textbf{Ingress}: Roteamento HTTP
            \end{itemize}
        \end{column}
        \begin{column}{0.5\textwidth}
            \textbf{Benefícios:}
            \begin{itemize}
                \item Auto-scaling
                \item Auto-healing
                \item Rolling updates
                \item Service discovery
                \item Load balancing
            \end{itemize}
        \end{column}
    \end{columns}
\end{frame}

\begin{frame}[fragile]
    \frametitle{Deployment Kubernetes}
    \begin{lstlisting}[language=bash]
apiVersion: apps/v1
kind: Deployment
metadata:
  name: user-service
  labels:
    app: user-service
spec:
  replicas: 3
  selector:
    matchLabels:
      app: user-service
  template:
    metadata:
      labels:
        app: user-service
    spec:
      containers:
      - name: user-service
        image: user-service:latest
        ports:
        - containerPort: 8000
        env:
        - name: DATABASE_URL
          value: "postgresql://user:pass@db:5432/users"
---
apiVersion: v1
kind: Service
metadata:
  name: user-service
spec:
  selector:
    app: user-service
  ports:
  - port: 80
    targetPort: 8000
  type: ClusterIP
    \end{lstlisting}
\end{frame}

\section{Demonstração Prática}

\begin{frame}
    \frametitle{Demonstração: API Completa}
    \begin{enumerate}
        \item \textbf{Criar API FastAPI} (10 min)
        \begin{itemize}
            \item Estrutura básica
            \item Endpoints CRUD
            \item Validação de dados
        \end{itemize}
        
        \item \textbf{Containerizar com Docker} (10 min)
        \begin{itemize}
            \item Dockerfile
            \item Build da imagem
            \item Executar container
        \end{itemize}
        
        \item \textbf{Microservices com Docker Compose} (15 min)
        \begin{itemize}
            \item Múltiplos serviços
            \item Comunicação entre serviços
            \item Banco de dados
        \end{itemize}
        
        \item \textbf{Deploy no Kubernetes} (15 min)
        \begin{itemize}
            \item Configuração YAML
            \item Deploy dos serviços
            \item Teste da aplicação
        \end{itemize}
        
        \item \textbf{Testes e Monitoramento} (10 min)
        \begin{itemize}
            \item Testes de carga
            \item Logs e métricas
            \item Escalabilidade
        \end{itemize}
    \end{enumerate}
\end{frame}

\section{Próximos Passos}

\begin{frame}
    \frametitle{O que aprender depois?}
    \begin{columns}
        \begin{column}{0.5\textwidth}
            \textbf{Conceitos Avançados:}
            \begin{itemize}
                \item Service Mesh (Istio)
                \item Message Queues
                \item Event Sourcing
                \item CQRS Pattern
                \item Circuit Breaker
                \item Distributed Tracing
            \end{itemize}
        \end{column}
        \begin{column}{0.5\textwidth}
            \textbf{Ferramentas:}
            \begin{itemize}
                \item Prometheus + Grafana
                \item ELK Stack
                \item Jaeger
                \item Helm
                \item ArgoCD
                \item Terraform
            \end{itemize}
        \end{column}
    \end{columns}
    
    \vspace{1em}
    \begin{alertblock}{Recursos para Estudo}
        \begin{itemize}
            \item \href{https://fastapi.tiangolo.com/}{Documentação FastAPI}
            \item \href{https://kubernetes.io/docs/}{Documentação Kubernetes}
            \item \href{https://docs.docker.com/}{Documentação Docker}
            \item \href{https://microservices.io/}{Microservices Patterns}
        \end{itemize}
    \end{alertblock}
\end{frame}

\begin{frame}
    \frametitle{Conclusão}
    \begin{center}
        \Huge{Obrigado!}
        
        \vspace{1em}
        \Large{Perguntas?}
        
        \vspace{2em}
        \normalsize
        \textbf{Contato:} bento.siqueira@ufla.br
        
        \textbf{Repositório:} github.com/californi/disciplinas-ufla
    \end{center}
\end{frame}

\end{document}
